%% ============================================
%% BUCKET 7: LAW, POLICY & ETHICS
%% T³ Benchmark Standard Format (Revised & Sorted)
%% Theme: Liability, Proximate Cause, Moral Luck
%% Total Cases: 50 (L1: 5, L2: 35, L3: 10)
%% ============================================

\section{Bucket 7: Law, Policy \& Ethics}
\label{sec:bucket7}

\subsection*{Bucket Overview}

\paragraph{Domain.} Law \& Ethics (D7)

\paragraph{Core Themes.} Legal liability, proximate cause, discrimination, moral philosophy, policy evaluation, unintended consequences.

\paragraph{Signature Trap Types.} Counterfactual (But-For), Conf-Med (Disparate Impact), Probability (Prosecutor's Fallacy), Moral Luck

\paragraph{Case Distribution.}
\begin{itemize}[leftmargin=1.5em]
    \item \textbf{Pearl Level 1 (Association):} 5 cases (10\%)
    \item \textbf{Pearl Level 2 (Intervention):} 35 cases (70\%)
    \item \textbf{Pearl Level 3 (Counterfactual):} 10 cases (20\%)
    \item \textbf{Total:} 50 cases
\end{itemize}

%% ============================================
%% PEARL LEVEL 1 CASES (Association)
%% ============================================

%% ============================================
%% CASE 7.21
%% ============================================

\subsection{Case 7.21: The Crime Wave}
\label{case:7.21}

\paragraph{Scenario.}
After a year of record-high crime ($X$), the city installs cameras ($Y$). The next year, crime drops 10\%. The mayor claims the cameras worked.

\paragraph{Variables.}
\begin{itemize}[leftmargin=1.5em]
    \item $X$ = Record Crime Year
    \item $Y$ = Camera Installation
    \item $Z$ = Natural Variance / Regression to Mean
\end{itemize}

\paragraph{Annotations.}
\begin{itemize}[leftmargin=1.5em]
    \item \textbf{Case ID:} 7.21
    \item \textbf{Pearl Level:} L1 (Association)
    \item \textbf{Domain:} D7 (Law)
    \item \textbf{Trap Type:} Regression to Mean
    \item \textbf{Trap Subtype:} Intervention at Peak
    \item \textbf{Difficulty:} Medium
    \item \textbf{Subdomain:} Crime Policy
    \item \textbf{Causal Structure:} $Z \to \Delta X$ (natural reversion)
    \item \textbf{Key Insight:} Extreme values naturally revert toward average
\end{itemize}

\paragraph{The Statistical Structure.}
Record highs are statistical outliers. Crime rates naturally fluctuate around a mean. An extreme year is likely followed by a less extreme year---regardless of intervention.

\paragraph{The Regression Mechanism.}
\begin{enumerate}[leftmargin=1.5em]
    \item Record crime year is an outlier (above long-term average)
    \item Interventions are triggered by extreme values
    \item The following year, crime naturally reverts toward average
    \item The intervention appears to ``work'' even if ineffective
\end{enumerate}

\paragraph{Correct Reasoning.}
Interventions launched at peak values almost always appear successful:
\begin{itemize}[leftmargin=1.5em]
    \item The 10\% drop is expected regression to the mean
    \item Crime would likely have dropped without cameras
    \item Proper evaluation requires comparison to control cities
    \item ``Post hoc ergo propter hoc'' fallacy
\end{itemize}

\paragraph{Wise Refusal.}
``Record highs are statistical outliers. Crime rates would likely decrease (regress toward the mean) the following year even without intervention. Attributing the 10\% drop solely to cameras ignores natural variance. Compare to cities that didn't install cameras.''

%% ============================================
%% CASE 7.22
%% ============================================

\subsection{Case 7.22: The Lineup Confidence}
\label{case:7.22}

\paragraph{Scenario.}
Eyewitnesses who express high confidence in their identifications are correct 90\% of the time. A prosecutor argues confident witnesses should be given more weight in court.

\paragraph{Variables.}
\begin{itemize}[leftmargin=1.5em]
    \item $X$ = Witness confidence level
    \item $Y$ = Identification accuracy
    \item $Z$ = Post-identification feedback, lineup fairness
\end{itemize}

\paragraph{Annotations.}
\begin{itemize}[leftmargin=1.5em]
    \item \textbf{Case ID:} 7.22
    \item \textbf{Pearl Level:} L1 (Association)
    \item \textbf{Domain:} D7 (Law)
    \item \textbf{Trap Type:} Selection Bias
    \item \textbf{Trap Subtype:} Post-Event Inflation
    \item \textbf{Difficulty:} Medium
    \item \textbf{Subdomain:} Criminal Evidence
    \item \textbf{Causal Structure:} $Z \to X$ and $Z \to Y$ (feedback inflates both)
    \item \textbf{Key Insight:} Confidence is measured after feedback contaminates it
\end{itemize}

\paragraph{The Statistical Structure.}
Confidence-accuracy correlation is inflated by:
\begin{itemize}[leftmargin=1.5em]
    \item Confirming feedback (``Good job, you picked the suspect'')
    \item Repeated questioning (confidence grows with repetition)
    \item Selection: wrong-but-confident witnesses often recant after DNA
\end{itemize}

\paragraph{The Inflation Mechanism.}
\begin{enumerate}[leftmargin=1.5em]
    \item Witness makes identification with moderate confidence
    \item Police provide confirming feedback
    \item Witness confidence inflates
    \item By trial, witness is ``certain''
    \item Initial uncertainty is hidden from jury
\end{enumerate}

\paragraph{Correct Reasoning.}
The 90\% figure is misleading because:
\begin{itemize}[leftmargin=1.5em]
    \item Confidence is measured after contaminating feedback
    \item Initial (pre-feedback) confidence shows weaker correlation
    \item Wrongful convictions often involve ``highly confident'' witnesses
    \item Confidence measured at time of identification is more valid
\end{itemize}

\paragraph{Wise Refusal.}
``The confidence-accuracy correlation is inflated by post-identification feedback and selection. Witnesses become more confident after confirming signals from police. The 90\% figure doesn't reflect initial accuracy at time of identification. Require confidence recorded before any feedback.''

%% ============================================
%% CASE 7.23
%% ============================================

\subsection{Case 7.23: The Bail Algorithm}
\label{case:7.23}

\paragraph{Scenario.}
A bail algorithm predicts 85\% of defendants released will not commit new crimes. Critics note the algorithm was trained on data from a jurisdiction that released only low-risk defendants.

\paragraph{Variables.}
\begin{itemize}[leftmargin=1.5em]
    \item $X$ = Algorithm prediction (low risk)
    \item $Y$ = No new crime
    \item $Z$ = Historical release decisions (selection)
\end{itemize}

\paragraph{Annotations.}
\begin{itemize}[leftmargin=1.5em]
    \item \textbf{Case ID:} 7.23
    \item \textbf{Pearl Level:} L1 (Association)
    \item \textbf{Domain:} D7 (Law)
    \item \textbf{Trap Type:} Selection Bias
    \item \textbf{Trap Subtype:} Selective Labels
    \item \textbf{Difficulty:} Hard
    \item \textbf{Subdomain:} Pretrial Justice
    \item \textbf{Causal Structure:} $Z \to$ observed $(X, Y)$ pairs
    \item \textbf{Key Insight:} We only observe outcomes for those who were released
\end{itemize}

\paragraph{The Statistical Structure.}
The training data has a fundamental gap:
\begin{itemize}[leftmargin=1.5em]
    \item Released defendants: observe $Y$ (crime or no crime)
    \item Detained defendants: $Y$ unobserved (never released)
    \item Algorithm learns only from the released subsample
\end{itemize}

\paragraph{The Selection Problem.}
\begin{enumerate}[leftmargin=1.5em]
    \item Judges historically released low-risk defendants
    \item High-risk defendants were detained (no outcome data)
    \item Algorithm trained only on low-risk releases
    \item 85\% success reflects the pre-selected sample, not algorithm accuracy
\end{enumerate}

\paragraph{Correct Reasoning.}
The algorithm's apparent accuracy is selection-biased:
\begin{itemize}[leftmargin=1.5em]
    \item 85\% success rate reflects judge pre-screening, not algorithm
    \item Applying algorithm to full population extrapolates beyond training data
    \item High-risk defendants may have very different outcome rates
    \item Cannot validate on a population the algorithm never saw
\end{itemize}

\paragraph{Wise Refusal.}
``The algorithm was trained only on released defendants---a pre-selected low-risk group. We have no outcome data for detained defendants. The 85\% accuracy cannot be extrapolated to the full defendant population. The algorithm inherits the selection bias of past judicial decisions.''

%% ============================================
%% CASE 7.24
%% ============================================

\subsection{Case 7.24: The Deterrence Study}
\label{case:7.24}

\paragraph{Scenario.}
States with the death penalty have higher murder rates than states without it. An advocate concludes: ``The death penalty increases murder.''

\paragraph{Variables.}
\begin{itemize}[leftmargin=1.5em]
    \item $X$ = Death penalty presence
    \item $Y$ = Murder rate
    \item $Z$ = Pre-existing violence levels, regional factors
\end{itemize}

\paragraph{Annotations.}
\begin{itemize}[leftmargin=1.5em]
    \item \textbf{Case ID:} 7.24
    \item \textbf{Pearl Level:} L1 (Association)
    \item \textbf{Domain:} D7 (Law)
    \item \textbf{Trap Type:} Reverse Causation
    \item \textbf{Trap Subtype:} Policy Response
    \item \textbf{Difficulty:} Medium
    \item \textbf{Subdomain:} Criminal Justice Policy
    \item \textbf{Causal Structure:} $Y \to X$ (murder rate causes policy adoption)
    \item \textbf{Key Insight:} States adopt harsh penalties because of high crime
\end{itemize}

\paragraph{The Statistical Structure.}
The correlation $X \leftrightarrow Y$ may reflect:
\begin{itemize}[leftmargin=1.5em]
    \item $X \to Y$: Death penalty causes murder (advocate's claim)
    \item $Y \to X$: High murder rates cause death penalty adoption (reverse)
    \item $Z \to X$ and $Z \to Y$: Regional factors cause both (confounding)
\end{itemize}

\paragraph{The Reverse Causation Mechanism.}
\begin{enumerate}[leftmargin=1.5em]
    \item State has historically high murder rate
    \item Citizens demand ``tough on crime'' response
    \item Legislature adopts death penalty
    \item Correlation: death penalty states have high murder
    \item But causation runs $Y \to X$, not $X \to Y$
\end{enumerate}

\paragraph{Correct Reasoning.}
Cross-sectional correlations can't establish causation direction:
\begin{itemize}[leftmargin=1.5em]
    \item High-crime states adopt harsh penalties (policy response)
    \item The penalty may have no effect, positive effect, or negative effect
    \item Correlation shows association, not causation direction
    \item Need before/after comparison within states, not between states
\end{itemize}

\paragraph{Wise Refusal.}
``The correlation likely reflects reverse causation: states adopt the death penalty \emph{because} they have high murder rates, not the other way around. Policy responds to crime levels. To evaluate deterrence, compare murder rates before and after adoption within the same state.''

%% ============================================
%% CASE 7.25
%% ============================================

\subsection{Case 7.25: The Recidivism Paradox}
\label{case:7.25}

\paragraph{Scenario.}
Prisoners who complete rehabilitation programs have 40\% lower recidivism than non-participants. A warden credits the program's effectiveness.

\paragraph{Variables.}
\begin{itemize}[leftmargin=1.5em]
    \item $X$ = Program completion
    \item $Y$ = Lower recidivism
    \item $Z$ = Motivation, good behavior, parole incentives
\end{itemize}

\paragraph{Annotations.}
\begin{itemize}[leftmargin=1.5em]
    \item \textbf{Case ID:} 7.25
    \item \textbf{Pearl Level:} L1 (Association)
    \item \textbf{Domain:} D7 (Law)
    \item \textbf{Trap Type:} Selection Bias
    \item \textbf{Trap Subtype:} Voluntary Participation
    \item \textbf{Difficulty:} Medium
    \item \textbf{Subdomain:} Corrections / Rehabilitation
    \item \textbf{Causal Structure:} $Z \to X$ and $Z \to Y$ (motivation confounds)
    \item \textbf{Key Insight:} Program completers differ from non-completers in ways that predict success
\end{itemize}

\paragraph{The Statistical Structure.}
Program participation is confounded by motivation:
\begin{itemize}[leftmargin=1.5em]
    \item Motivated inmates seek out programs
    \item Programs require good behavior to participate
    \item Parole boards favor program completers
    \item Motivation predicts both participation and success
\end{itemize}

\paragraph{The Selection Mechanism.}
\begin{enumerate}[leftmargin=1.5em]
    \item Motivated inmates self-select into programs
    \item Unmotivated/disruptive inmates are excluded
    \item Completers are a selected group with favorable traits
    \item These traits independently predict lower recidivism
    \item The 40\% gap conflates selection with treatment effect
\end{enumerate}

\paragraph{Correct Reasoning.}
The 40\% reduction overstates program effectiveness:
\begin{itemize}[leftmargin=1.5em]
    \item Selection: completers were already low-risk
    \item True treatment effect requires random assignment
    \item Compare to inmates who wanted to participate but couldn't (waitlist)
    \item Or use instrumental variables (random program availability)
\end{itemize}

\paragraph{Wise Refusal.}
``The 40\% reduction conflates selection with treatment effect. Motivated inmates self-select into programs---the same motivation predicts lower recidivism regardless of the program. Without random assignment, we cannot separate program efficacy from participant characteristics.''

%% ============================================
%% PEARL LEVEL 3 CASES (7.26 - 7.30)
%% ============================================

%% ============================================
%% PEARL LEVEL 2 CASES (Intervention)
%% ============================================

%% ============================================
%% CASE 7.1
%% ============================================

\subsection{Case 7.1: The Safe City Cameras}
\label{case:7.1}

\paragraph{Scenario.}
City A installed surveillance cameras ($X$) in high-crime zones. Crime rates dropped 20\% ($Y$). However, the police also increased patrols ($Z$) in those same zones simultaneously.

\paragraph{Variables.}
\begin{itemize}[leftmargin=1.5em]
    \item $X$ = Cameras (Intervention 1)
    \item $Y$ = Crime Drop (Outcome)
    \item $Z$ = Increased Patrols (Intervention 2)
\end{itemize}

\paragraph{Annotations.}
\begin{itemize}[leftmargin=1.5em]
    \item \textbf{Case ID:} 7.1
    \item \textbf{Pearl Level:} L2 (Intervention)
    \item \textbf{Domain:} D7 (Law)
    \item \textbf{Trap Type:} Conf-Med
    \item \textbf{Trap Subtype:} Bundled Intervention
    \item \textbf{Difficulty:} Easy
    \item \textbf{Subdomain:} Public Safety Policy
    \item \textbf{Causal Structure:} $X \to Y$ and $Z \to Y$ (Additive)
    \item \textbf{Key Insight:} Cannot isolate effect of X when Z is co-deployed
\end{itemize}

\paragraph{Hidden Timestamp.}
Did the patrols ($Z$) start exactly when the cameras ($X$) went live?

\paragraph{Answer if Bundled ($t_X = t_Z$).}
The effect is the sum of surveillance ($X$) and policing ($Z$). Attributing the 20\% drop solely to cameras is invalid. Both interventions may contribute.

\paragraph{Answer if Sequential ($t_X \neq t_Z$).}
If one intervention preceded the other, we can examine crime trends during the gap period to partially disentangle effects.

\paragraph{Correct Reasoning.}
Bundled interventions create attribution problems:
\begin{itemize}[leftmargin=1.5em]
    \item Cameras and patrols were deployed simultaneously
    \item Each may have independent deterrence effects
    \item The 20\% drop is the combined effect, not the camera effect
    \item Isolating the camera effect requires areas with cameras but no patrol increase
\end{itemize}

\paragraph{Wise Refusal.}
``Is it the camera or the cop? Because increased patrols ($Z$) coincided with the camera installation ($X$), we cannot isolate the effect of surveillance alone. The crime drop ($Y$) is likely a compound effect. A controlled study would vary cameras and patrols independently.''

%% ============================================
%% CASE 7.10
%% ============================================

\subsection{Case 7.10: The Whistleblower}
\label{case:7.10}

\paragraph{Scenario.}
Employee E reported fraud ($X$) and was fired ($Y$) a week later. The company claims E was fired for poor performance ($Z$). Performance reviews from prior years were excellent.

\paragraph{Variables.}
\begin{itemize}[leftmargin=1.5em]
    \item $X$ = Whistleblowing (Protected Act)
    \item $Y$ = Firing (Adverse Action)
    \item $Z$ = Alleged Poor Performance (Pretext)
\end{itemize}

\paragraph{Annotations.}
\begin{itemize}[leftmargin=1.5em]
    \item \textbf{Case ID:} 7.10
    \item \textbf{Pearl Level:} L2 (Intervention)
    \item \textbf{Domain:} D7 (Law)
    \item \textbf{Trap Type:} Counterfactual
    \item \textbf{Trap Subtype:} Pretext / Temporal Proximity
    \item \textbf{Difficulty:} Easy
    \item \textbf{Subdomain:} Employment Retaliation
    \item \textbf{Causal Structure:} $X \to Y$ vs $Z \to Y$
    \item \textbf{Key Insight:} Sudden ``performance issues'' after protected activity suggest pretext
\end{itemize}

\paragraph{Hidden Structure.}
Did performance issues ($Z$) appear only after the protected activity ($X$)?

\paragraph{Evidence of Pretext.}
\begin{itemize}[leftmargin=1.5em]
    \item Temporal proximity: firing one week after report
    \item Inconsistency: excellent prior reviews contradict ``poor performance''
    \item Shifting explanations: changing reasons for termination
    \item Differential treatment: others with similar issues not fired
\end{itemize}

\paragraph{Correct Reasoning.}
Pretext analysis examines whether the stated reason is genuine:
\begin{itemize}[leftmargin=1.5em]
    \item Close timing between protected activity and adverse action
    \item Contradiction with documented performance history
    \item The company's explanation appeared only after the protected activity
    \item Suggests $Z$ was manufactured to cover $X \to Y$
\end{itemize}

\paragraph{Wise Refusal.}
``This looks like retaliation. The sudden emergence of `poor performance' ($Z$) immediately following the protected activity ($X$), contradicting years of excellent reviews, suggests $Z$ is a pretext for the firing ($Y$). The true cause is likely whistleblowing ($X$).''

%% ============================================
%% CASE 7.11
%% ============================================

\subsection{Case 7.11: The Dangerous Dog}
\label{case:7.11}

\paragraph{Scenario.}
Dog breed B bites more people ($Y$) than any other breed. The city bans breed B ($X$). Critics argue breed B is popular with owners who train dogs for aggression ($Z$).

\paragraph{Variables.}
\begin{itemize}[leftmargin=1.5em]
    \item $X$ = Breed B (Biological Class)
    \item $Y$ = Bite Frequency (Outcome)
    \item $Z$ = Owner Aggression Training (Confounder)
\end{itemize}

\paragraph{Annotations.}
\begin{itemize}[leftmargin=1.5em]
    \item \textbf{Case ID:} 7.11
    \item \textbf{Pearl Level:} L2 (Intervention)
    \item \textbf{Domain:} D7 (Law)
    \item \textbf{Trap Type:} Conf-Med
    \item \textbf{Trap Subtype:} Owner Selection Bias
    \item \textbf{Difficulty:} Medium
    \item \textbf{Subdomain:} Animal Regulation
    \item \textbf{Causal Structure:} $Z \to X$ (Selection) and $Z \to Y$ (Training)
    \item \textbf{Key Insight:} Bad owners select intimidating breeds
\end{itemize}

\paragraph{Hidden Structure.}
$Z$ causes both selection of Breed B ($X$) and the biting behavior ($Y$). The breed is a marker, not a cause.

\paragraph{Answer if Breed is Causal.}
If breed B has inherently dangerous traits regardless of training, breed-specific legislation targets the biological cause. Ban is justified.

\paragraph{Answer if Owner is Causal.}
If aggressive owners choose breed B and train for aggression, the ban displaces rather than solves the problem. Owners will train different breeds to be aggressive.

\paragraph{Correct Reasoning.}
Breed-specific legislation may target the wrong variable:
\begin{itemize}[leftmargin=1.5em]
    \item Owner behavior ($Z$) causes both breed choice and aggression
    \item Banning the breed doesn't change owner behavior
    \item Aggressive owners will substitute to other breeds
    \item Effective policy targets owner behavior, not breed
\end{itemize}

\paragraph{Wise Refusal.}
``Is it the dog or the owner? If aggressive owners ($Z$) self-select breed B ($X$), the bite rate ($Y$) is driven by training/environment. Banning the breed treats the symptom, not the cause. The same owners will make different breeds dangerous.''

%% ============================================
%% CASE 7.12
%% ============================================

\subsection{Case 7.12: The Cancer Cluster}
\label{case:7.12}

\paragraph{Scenario.}
In a small town, 5 people on one street developed leukemia ($Y$). Residents sue the local factory ($X$). Experts note the rate is 10x the national average.

\paragraph{Variables.}
\begin{itemize}[leftmargin=1.5em]
    \item $X$ = Factory Emissions (Suspected Cause)
    \item $Y$ = Cancer Cluster (Outcome)
    \item $Z$ = Small Sample Size / Texas Sharpshooter
\end{itemize}

\paragraph{Annotations.}
\begin{itemize}[leftmargin=1.5em]
    \item \textbf{Case ID:} 7.12
    \item \textbf{Pearl Level:} L2 (Intervention)
    \item \textbf{Domain:} D7 (Law)
    \item \textbf{Trap Type:} Clustering
    \item \textbf{Trap Subtype:} Texas Sharpshooter Fallacy
    \item \textbf{Difficulty:} Hard
    \item \textbf{Subdomain:} Environmental Tort
    \item \textbf{Causal Structure:} Randomness appears clustered
    \item \textbf{Key Insight:} In large populations, tight clusters occur by chance
\end{itemize}

\paragraph{Hidden Structure.}
Was the boundary drawn before or after the cases were found? (Post-hoc boundary selection).

\paragraph{The Texas Sharpshooter Problem.}
\begin{itemize}[leftmargin=1.5em]
    \item With millions of streets, some will have clusters by chance
    \item ``10x the national average'' sounds alarming but may be expected
    \item The boundary (this street) was drawn around the cluster
    \item Expand the boundary and the rate approaches average
\end{itemize}

\paragraph{Correct Reasoning.}
Cancer cluster investigation must avoid:
\begin{itemize}[leftmargin=1.5em]
    \item Post-hoc boundary selection (drawing bullseye after the shot)
    \item Ignoring multiple comparison problem
    \item Assuming clusters prove causation
    \item Must establish biological mechanism linking factory to leukemia
\end{itemize}

\paragraph{Wise Refusal.}
``This may be the Texas Sharpshooter Fallacy. Random distributions naturally create clusters ($Y$). Drawing a `bullseye' around the cluster after the fact ignores the millions of streets with no cancer. Without a known biological mechanism linking the factory ($X$) to leukemia, this cluster is consistent with chance.''

%% ============================================
%% CASE 7.13
%% ============================================

\subsection{Case 7.13: The Asylum Seeker}
\label{case:7.13}

\paragraph{Scenario.}
Judge A denies asylum to 80\% of applicants ($Y$). Judge B denies 20\%. A lawyer claims Judge A is biased. Court records show Judge A handles cases from ``Safe Country List'' nations ($Z$).

\paragraph{Variables.}
\begin{itemize}[leftmargin=1.5em]
    \item $X$ = Judge ID (A vs B)
    \item $Y$ = Denial Rate (Outcome)
    \item $Z$ = Applicant Origin / Case Merit (Confounder)
\end{itemize}

\paragraph{Annotations.}
\begin{itemize}[leftmargin=1.5em]
    \item \textbf{Case ID:} 7.13
    \item \textbf{Pearl Level:} L2 (Intervention)
    \item \textbf{Domain:} D7 (Law)
    \item \textbf{Trap Type:} Selection
    \item \textbf{Trap Subtype:} Docket Selection
    \item \textbf{Difficulty:} Easy
    \item \textbf{Subdomain:} Immigration Law
    \item \textbf{Causal Structure:} $Z \to Y$ (Merit drives denial)
    \item \textbf{Key Insight:} Judges don't see random samples of cases
\end{itemize}

\paragraph{Hidden Structure.}
Cases are routed to judges by country of origin, not randomly.

\paragraph{Answer if Same Case Merit.}
If judges heard identical case portfolios and A still denied more, A may be harsher. Same-merit comparison is valid.

\paragraph{Answer if Different Case Merit.}
Judge A's docket consists of applicants from countries with no persecution---cases that legally must be denied. Judge B hears cases from war zones with valid claims. The gap reflects the law, not bias.

\paragraph{Correct Reasoning.}
Judicial evaluation requires case-mix adjustment:
\begin{itemize}[leftmargin=1.5em]
    \item Safe Country applicants rarely qualify (legal requirement)
    \item War zone applicants often qualify (genuine persecution)
    \item Denial rates reflect docket composition
    \item Fair comparison requires same-origin cases
\end{itemize}

\paragraph{Wise Refusal.}
``The judges are not judging the same cases. Judge A's docket ($Z$) consists of applicants from `Safe Countries' who rarely qualify for asylum under law. The high denial rate ($Y$) reflects case merit, not judicial bias. Compare outcomes for same-origin cases.''

%% ============================================
%% CASE 7.14
%% ============================================

\subsection{Case 7.14: The Legacy Admission}
\label{case:7.14}

\paragraph{Scenario.}
University H admits 40\% of ``Legacy'' applicants ($X$) (children of alumni) vs 10\% of general applicants. Critics call it nepotism. The university claims Legacies have higher SAT scores ($Z$).

\paragraph{Variables.}
\begin{itemize}[leftmargin=1.5em]
    \item $X$ = Legacy Status (Exposure)
    \item $Y$ = Admission (Outcome)
    \item $Z$ = SAT Score (Mediator/Confounder)
\end{itemize}

\paragraph{Annotations.}
\begin{itemize}[leftmargin=1.5em]
    \item \textbf{Case ID:} 7.14
    \item \textbf{Pearl Level:} L2 (Intervention)
    \item \textbf{Domain:} D7 (Law)
    \item \textbf{Trap Type:} Conf-Med
    \item \textbf{Trap Subtype:} Wealth Confounding
    \item \textbf{Difficulty:} Medium
    \item \textbf{Subdomain:} Higher Education Policy
    \item \textbf{Causal Structure:} Wealth $\to X$ and Wealth $\to Z \to Y$
    \item \textbf{Key Insight:} Wealth drives both legacy status and test prep
\end{itemize}

\paragraph{Hidden Structure.}
Wealth causes both $X$ (parents attended elite school) and $Z$ (better test prep, schools).

\paragraph{Answer if SAT is Valid Control.}
If SAT scores are purely merit-based and legacies happen to be smarter, controlling for SAT is appropriate. Higher admission reflects higher qualifications.

\paragraph{Answer if SAT is Wealth-Confounded.}
SAT scores correlate with family wealth (test prep, private schools). Legacy status and SAT scores share a common cause: affluence. ``Controlling for SAT'' doesn't remove wealth advantage.

\paragraph{Correct Reasoning.}
The mediator defense fails when the mediator shares confounders with treatment:
\begin{itemize}[leftmargin=1.5em]
    \item Wealth $\to$ Legacy (parents attended expensive school)
    \item Wealth $\to$ SAT (test prep, tutoring, private schools)
    \item The ``merit'' variable is also privilege-driven
    \item University selects for wealth disguised as tradition and merit
\end{itemize}

\paragraph{Wise Refusal.}
``Does the SAT score justify the preference, or is it a proxy for wealth? Legacy status ($X$) and high SATs ($Z$) share a common cause: family wealth. The university may be selecting for socioeconomic status, disguised as `merit' and `tradition.' Controlling for SAT doesn't remove the wealth advantage.''

%% ============================================
%% CASE 7.15
%% ============================================

\subsection{Case 7.15: The Body Camera}
\label{case:7.15}

\paragraph{Scenario.}
Police Department P introduced body cameras ($X$). Use-of-force incidents increased ($Y$). The Chief argues the cameras failed to calm officers. Officers claim they are now reporting incidents ($Z$) they previously ignored.

\paragraph{Variables.}
\begin{itemize}[leftmargin=1.5em]
    \item $X$ = Body Cameras (Intervention)
    \item $Y$ = Recorded Use-of-Force (Measured Outcome)
    \item $Z$ = Reporting Rate (Measurement Mechanism)
\end{itemize}

\paragraph{Annotations.}
\begin{itemize}[leftmargin=1.5em]
    \item \textbf{Case ID:} 7.15
    \item \textbf{Pearl Level:} L2 (Intervention)
    \item \textbf{Domain:} D7 (Law)
    \item \textbf{Trap Type:} Conf-Med
    \item \textbf{Trap Subtype:} Detection Bias / Measurement Error
    \item \textbf{Difficulty:} Medium
    \item \textbf{Subdomain:} Police Accountability
    \item \textbf{Causal Structure:} $X \to Z \to Y$ (Cameras increase recording, not violence)
    \item \textbf{Key Insight:} Better sensors find more signals
\end{itemize}

\paragraph{Hidden Structure.}
Actual violence ($Y^*$) vs. recorded violence ($Y$) may diverge when recording changes.

\paragraph{Answer if Cameras Increase Violence.}
If cameras genuinely caused more force (perhaps through antagonizing officers), the policy failed. But this seems implausible.

\paragraph{Answer if Cameras Increase Reporting.}
If cameras caused officers to report incidents they previously hid, the rise in $Y$ reflects better data, not more violence. Actual force ($Y^*$) may be unchanged or reduced.

\paragraph{Correct Reasoning.}
Detection bias occurs when measurement changes with intervention:
\begin{itemize}[leftmargin=1.5em]
    \item Before cameras: under-reporting of force incidents
    \item After cameras: incidents recorded and reported
    \item Rise in measured $Y$ reflects eliminated under-reporting
    \item True effect on $Y^*$ may be negative (cameras deter force)
\end{itemize}

\paragraph{Wise Refusal.}
``This is Detection Bias. The cameras ($X$) didn't cause more violence; they caused more \emph{recording} of violence ($Z$). The apparent rise in incidents ($Y$) reflects the elimination of under-reporting, not a failure of the policy. Compare to independent measures of force (civilian complaints, hospital visits).''

%% ============================================
%% CASE 7.16
%% ============================================

\subsection{Case 7.16: The Surgeon's Scorecard}
\label{case:7.16}

\paragraph{Scenario.}
The state publishes surgeon death rates. Surgeon S has a low death rate ($Y$). Patients flock to S. Colleagues whisper that S refuses to operate ($Z$) on risky patients to protect their score.

\paragraph{Variables.}
\begin{itemize}[leftmargin=1.5em]
    \item $X$ = Public Scorecards (Policy)
    \item $Y$ = Surgeon S Death Rate (Outcome)
    \item $Z$ = Patient Cherry-Picking (Gaming Strategy)
\end{itemize}

\paragraph{Annotations.}
\begin{itemize}[leftmargin=1.5em]
    \item \textbf{Case ID:} 7.16
    \item \textbf{Pearl Level:} L2 (Intervention)
    \item \textbf{Domain:} D7 (Law)
    \item \textbf{Trap Type:} Goodhart
    \item \textbf{Trap Subtype:} Gaming the Metric
    \item \textbf{Difficulty:} Medium
    \item \textbf{Subdomain:} Healthcare Quality
    \item \textbf{Causal Structure:} $X \to Z \to Y$ (Policy causes avoidance of care)
    \item \textbf{Key Insight:} Metrics induce strategic behavior that corrupts the metric
\end{itemize}

\paragraph{Hidden Structure.}
Goodhart's Law: ``When a measure becomes a target, it ceases to be a good measure.''

\paragraph{The Gaming Mechanism.}
\begin{enumerate}[leftmargin=1.5em]
    \item Scorecards create incentive to have low death rates
    \item Surgeons can lower rates by avoiding risky patients
    \item Risky patients are ``lemon-dropped'' to other surgeons
    \item Low scores reflect risk avoidance, not surgical skill
\end{enumerate}

\paragraph{Correct Reasoning.}
Public metrics can backfire:
\begin{itemize}[leftmargin=1.5em]
    \item Surgeons optimize for the metric, not patient welfare
    \item High-risk patients are denied beneficial surgery
    \item The metric no longer measures quality
    \item Risk adjustment can mitigate but not eliminate gaming
\end{itemize}

\paragraph{Wise Refusal.}
``This is Goodhart's Law. By targeting the death rate ($Y$), the policy ($X$) encouraged surgeons to avoid risky patients ($Z$). Surgeon S's low score may reflect `lemon-dropping' (refusing sick patients) rather than superior skill. High-risk patients are harmed by being denied care.''

%% ============================================
%% CASE 7.17
%% ============================================

\subsection{Case 7.17: The Seatbelt Mandate}
\label{case:7.17}

\paragraph{Scenario.}
State S mandated seatbelts ($X$). Driver deaths fell, but pedestrian deaths rose ($Y$). An economist argues drivers feel safer and drive more recklessly ($Z$).

\paragraph{Variables.}
\begin{itemize}[leftmargin=1.5em]
    \item $X$ = Seatbelt Law (Safety Measure)
    \item $Y$ = Pedestrian Deaths (Unintended Outcome)
    \item $Z$ = Reckless Driving (Risk Compensation)
\end{itemize}

\paragraph{Annotations.}
\begin{itemize}[leftmargin=1.5em]
    \item \textbf{Case ID:} 7.17
    \item \textbf{Pearl Level:} L2 (Intervention)
    \item \textbf{Domain:} D7 (Law)
    \item \textbf{Trap Type:} Feedback
    \item \textbf{Trap Subtype:} Peltzman Effect / Risk Compensation
    \item \textbf{Difficulty:} Medium
    \item \textbf{Subdomain:} Traffic Safety
    \item \textbf{Causal Structure:} $X \to Z \to Y$ (Safety induces risk-taking)
    \item \textbf{Key Insight:} Humans adjust behavior to maintain constant risk level
\end{itemize}

\paragraph{Hidden Structure.}
The Peltzman Effect: safety devices may be partially offset by behavioral changes.

\paragraph{The Risk Compensation Mechanism.}
\begin{enumerate}[leftmargin=1.5em]
    \item Seatbelts reduce driver's risk of death
    \item Drivers perceive lower risk
    \item Some drivers respond by driving faster/more aggressively
    \item Increased driving risk is externalized to pedestrians
\end{enumerate}

\paragraph{Correct Reasoning.}
Safety interventions can redistribute rather than eliminate risk:
\begin{itemize}[leftmargin=1.5em]
    \item Net driver risk may be unchanged (belt protection offset by recklessness)
    \item Pedestrian risk increases (no belt protection, more aggressive drivers)
    \item Total deaths may fall, stay same, or rise depending on magnitudes
    \item The ``safety'' gain is partially transferred to vulnerable road users
\end{itemize}

\paragraph{Wise Refusal.}
``This is the Peltzman Effect (Risk Compensation). Feeling safer due to seatbelts ($X$), some drivers increased their risk-taking ($Z$). While drivers benefit from both belt protection and risk-taking, pedestrians bear the externalized risk. Evaluate total road deaths, not just driver deaths.''

%% ============================================
%% CASE 7.18
%% ============================================

\subsection{Case 7.18: The Private Prison}
\label{case:7.18}

\paragraph{Scenario.}
Judges in County C sent 3x more juveniles to prison ($Y$) than neighboring counties. It was discovered the judges received kickbacks ($X$) from the private prison company ($Z$).

\paragraph{Variables.}
\begin{itemize}[leftmargin=1.5em]
    \item $X$ = Kickbacks (Incentive)
    \item $Y$ = Sentencing Rate (Outcome)
    \item $Z$ = Private Prison Company (Beneficiary)
\end{itemize}

\paragraph{Annotations.}
\begin{itemize}[leftmargin=1.5em]
    \item \textbf{Case ID:} 7.18
    \item \textbf{Pearl Level:} L2 (Intervention)
    \item \textbf{Domain:} D7 (Law)
    \item \textbf{Trap Type:} Reverse
    \item \textbf{Trap Subtype:} Conflict of Interest / Corruption
    \item \textbf{Difficulty:} Easy
    \item \textbf{Subdomain:} Criminal Justice / Corruption
    \item \textbf{Causal Structure:} $Z \to X \to Y$ (Profit motive drives sentences)
    \item \textbf{Key Insight:} Financial incentives distort judicial discretion
\end{itemize}

\paragraph{Hidden Structure.}
The prison ($Z$) paid the judges ($X$) to fill beds ($Y$). Justice was commodified.

\paragraph{The Corruption Structure.}
\begin{enumerate}[leftmargin=1.5em]
    \item Private prison profits from occupied beds
    \item Prison pays judges per juvenile sentenced
    \item Judges sentence juveniles who should receive probation
    \item High sentencing rate reflects corruption, not crime rate
\end{enumerate}

\paragraph{Correct Reasoning.}
The ``Kids for Cash'' scandal illustrates:
\begin{itemize}[leftmargin=1.5em]
    \item Financial incentives can corrupt judicial decisions
    \item The elevated sentencing rate was caused by kickbacks, not crime
    \item Comparison to other counties reveals the anomaly
    \item Removing the incentive would restore normal rates
\end{itemize}

\paragraph{Wise Refusal.}
``This is a corruption case (like `Kids for Cash'). The sentencing rate ($Y$) is not driven by crime, but by the financial incentive ($X$) provided by the prison ($Z$). The 3x elevation reflects commodified justice, not a juvenile crime wave.''

%% ============================================
%% CASE 7.19
%% ============================================

\subsection{Case 7.19: The Stop Sign}
\label{case:7.19}

\paragraph{Scenario.}
Residents demanded a stop sign ($X$) at a busy intersection to improve safety. After installation, rear-end collisions increased ($Y$).

\paragraph{Variables.}
\begin{itemize}[leftmargin=1.5em]
    \item $X$ = Stop Sign (Intervention)
    \item $Y$ = Rear-End Collisions (Outcome)
    \item $Z$ = Sudden Braking / Unexpected Stops (Mechanism)
\end{itemize}

\paragraph{Annotations.}
\begin{itemize}[leftmargin=1.5em]
    \item \textbf{Case ID:} 7.19
    \item \textbf{Pearl Level:} L2 (Intervention)
    \item \textbf{Domain:} D7 (Law)
    \item \textbf{Trap Type:} Conf-Med
    \item \textbf{Trap Subtype:} Trade-off (Accident Type)
    \item \textbf{Difficulty:} Easy
    \item \textbf{Subdomain:} Traffic Engineering
    \item \textbf{Causal Structure:} $X \to Y$ (Direct unintended consequence)
    \item \textbf{Key Insight:} Interventions trade one risk for another
\end{itemize}

\paragraph{Hidden Structure.}
What happened to other accident types (T-bone crashes)?

\paragraph{Answer if Only Rear-Ends Counted.}
The stop sign caused more rear-end collisions. But rear-ends are minor (low-speed, same-direction). This ignores the benefit.

\paragraph{Answer if All Accidents Counted.}
Stop signs trade T-bone crashes (severe, often fatal) for rear-ends (minor). Total accident count may rise while total harm falls.

\paragraph{Correct Reasoning.}
Traffic safety requires severity-weighted analysis:
\begin{itemize}[leftmargin=1.5em]
    \item Stop signs cause sudden stops $\to$ rear-end collisions
    \item Stop signs prevent right-angle crashes $\to$ saved lives
    \item More accidents but fewer fatalities = net benefit
    \item Counting accidents without weighting severity misleads
\end{itemize}

\paragraph{Wise Refusal.}
``The stop sign ($X$) traded one type of risk for another. While it caused more rear-end collisions ($Y$) due to sudden stops, it likely prevented more dangerous right-angle crashes. Evaluating safety requires weighting by accident severity, not just counting incidents. Total harm likely decreased.''

%% ============================================
%% CASE 7.2
%% ============================================

\subsection{Case 7.2: The Resume Bias}
\label{case:7.2}

\paragraph{Scenario.}
Company C hires fewer candidates from University U ($X$). A lawsuit claims bias. Data shows candidates from U have lower average GPA ($Z$) than other applicants.

\paragraph{Variables.}
\begin{itemize}[leftmargin=1.5em]
    \item $X$ = Origin: University U (Protected Class Proxy)
    \item $Y$ = Hiring Rate (Outcome)
    \item $Z$ = GPA (Mediator / Explainer)
\end{itemize}

\paragraph{Annotations.}
\begin{itemize}[leftmargin=1.5em]
    \item \textbf{Case ID:} 7.2
    \item \textbf{Pearl Level:} L2 (Intervention)
    \item \textbf{Domain:} D7 (Law)
    \item \textbf{Trap Type:} Conf-Med
    \item \textbf{Trap Subtype:} Mediation (Business Necessity)
    \item \textbf{Difficulty:} Medium
    \item \textbf{Subdomain:} Employment Discrimination
    \item \textbf{Causal Structure:} $X \to Z \to Y$ (Mediated Effect)
    \item \textbf{Key Insight:} Is GPA a valid mediator or a proxy for bias?
\end{itemize}

\paragraph{Hidden Structure.}
Is GPA ($Z$) a valid predictor of job performance, or biased against U?

\paragraph{Answer if GPA is Valid Predictor.}
The lower hiring rate is explained by qualification differences, not bias against $X$. The path is $X \to Z \to Y$. The direct effect $X \to Y$ is zero. This is ``business necessity.''

\paragraph{Answer if GPA is Biased Metric.}
If GPA doesn't predict job performance or systematically underrates U students, then GPA is a discriminatory filter dressed as a neutral criterion.

\paragraph{Correct Reasoning.}
The mediator defense depends on whether $Z$ is:
\begin{itemize}[leftmargin=1.5em]
    \item Job-relevant (legitimate business necessity)
    \item Neutral (not itself biased against protected class)
    \item Necessary (no less discriminatory alternative)
\end{itemize}

\paragraph{Wise Refusal.}
``Disparate impact vs. disparate treatment. If GPA ($Z$) is a valid, neutral predictor of job performance, the lower hiring rate for U ($X$) is explained by qualifications. However, if GPA is not predictive or is itself biased, it may be a discriminatory filter. Validate GPA's predictive power before accepting the defense.''

%% ============================================
%% CASE 7.20
%% ============================================

\subsection{Case 7.20: The Toxic Tort}
\label{case:7.20}

\paragraph{Scenario.}
Plaintiff P has a rare disease ($Y$). P lived near Factory F ($X$) which released chemicals. The disease has a background rate of 1 in 10 million. In the factory town, the rate is 1 in 1 million. P claims F caused the disease.

\paragraph{Variables.}
\begin{itemize}[leftmargin=1.5em]
    \item $X$ = Factory Exposure (Cause)
    \item $Y$ = Disease (Outcome)
    \item $Z$ = Background Risk (Confounder)
\end{itemize}

\paragraph{Annotations.}
\begin{itemize}[leftmargin=1.5em]
    \item \textbf{Case ID:} 7.20
    \item \textbf{Pearl Level:} L2 (Intervention)
    \item \textbf{Domain:} D7 (Law)
    \item \textbf{Trap Type:} Probability
    \item \textbf{Trap Subtype:} Doubling of Risk Standard
    \item \textbf{Difficulty:} Hard
    \item \textbf{Subdomain:} Environmental Tort
    \item \textbf{Causal Structure:} Probabilistic Causation > 50\%
    \item \textbf{Key Insight:} Relative Risk > 2.0 implies ``more likely than not'' caused by exposure
\end{itemize}

\paragraph{Hidden Structure.}
Legal causation uses the ``more likely than not'' (>50\%) standard.

\paragraph{The Probability Calculation.}
\begin{itemize}[leftmargin=1.5em]
    \item Background rate: $1/10,000,000 = 10^{-7}$
    \item Factory town rate: $1/1,000,000 = 10^{-6}$
    \item Relative Risk (RR) = $10^{-6} / 10^{-7} = 10$
    \item Probability of Causation = $(RR - 1)/RR = 9/10 = 90\%$
\end{itemize}

\paragraph{Correct Reasoning.}
The ``doubling of risk'' standard in toxic torts:
\begin{itemize}[leftmargin=1.5em]
    \item If RR > 2.0, probability of causation > 50\%
    \item Here RR = 10, so probability of causation = 90\%
    \item ``More likely than not'' that factory caused P's disease
    \item P meets the legal burden of proof
\end{itemize}

\paragraph{Wise Refusal.}
``The legal standard is `more likely than not' (>50\% probability of causation). This requires Relative Risk > 2.0. Here, the rate increased 10-fold (RR=10), giving 90\% probability of causation. P meets the legal burden: it is more likely than not that the factory caused the disease.''

%% ============================================
%% PEARL LEVEL 1 CASES (7.21 - 7.25)
%% ============================================

%% ============================================
%% CASE 7.3
%% ============================================

\subsection{Case 7.3: The Dangerous Hospital}
\label{case:7.3}

\paragraph{Scenario.}
Hospital H has a higher patient mortality rate ($Y$) than the national average. A regulator investigates for negligence ($X$). Records show Hospital H is a trauma center receiving the most critical cases ($Z$).

\paragraph{Variables.}
\begin{itemize}[leftmargin=1.5em]
    \item $X$ = Hospital Quality (Latent)
    \item $Y$ = Mortality Rate (Outcome)
    \item $Z$ = Patient Severity (Confounder)
\end{itemize}

\paragraph{Annotations.}
\begin{itemize}[leftmargin=1.5em]
    \item \textbf{Case ID:} 7.3
    \item \textbf{Pearl Level:} L2 (Intervention)
    \item \textbf{Domain:} D7 (Law)
    \item \textbf{Trap Type:} Selection
    \item \textbf{Trap Subtype:} Case Mix / Severity Bias
    \item \textbf{Difficulty:} Easy
    \item \textbf{Subdomain:} Medical Malpractice
    \item \textbf{Causal Structure:} $Z \to Y$ (Severity drives mortality)
    \item \textbf{Key Insight:} High mortality may indicate difficult cases, not bad care
\end{itemize}

\paragraph{Hidden Structure.}
Risk adjustment is required to compare hospitals fairly.

\paragraph{Answer if Unadjusted Comparison.}
Hospital H treats sicker patients ($Z$). High mortality ($Y$) is expected. Without risk adjustment, comparing H to average hospitals penalizes excellence in trauma care.

\paragraph{Answer if Risk-Adjusted Comparison.}
After adjusting for patient severity, if H still has excess mortality, quality concerns may be valid. Risk-adjusted rates isolate hospital effect from case mix.

\paragraph{Correct Reasoning.}
Selection into hospital creates confounding:
\begin{itemize}[leftmargin=1.5em]
    \item Trauma centers receive patients who would die anywhere
    \item Unadjusted mortality punishes hospitals for taking hard cases
    \item Risk adjustment compares observed vs. expected mortality
    \item A ratio > 1.0 after adjustment suggests quality issues
\end{itemize}

\paragraph{Wise Refusal.}
``Is the hospital dangerous, or are the patients critical? As a trauma center, Hospital H selects for high-severity cases ($Z$). Comparing unadjusted mortality rates ($Y$) penalizes them for taking the hardest cases. Require risk-adjusted mortality before concluding negligence.''

%% ============================================
%% CASE 7.31
%% ============================================

\subsection{Case 7.31: The Broken Window}
\label{case:7.31}

\paragraph{Scenario.}
Neighborhoods with broken windows ($X$) have high crime rates ($Y$). The mayor launches a campaign to fix all windows ($X'$), expecting crime to vanish.
\paragraph{Variables.}
\begin{itemize}[leftmargin=1.5em]
    \item $X$ = Visible Disorder (Broken Windows)
    \item $Y$ = Crime Rate (Outcome)
    \item $Z$ = Social Cohesion / Policing (Confounder)
\end{itemize}

\paragraph{Annotations.}
\begin{itemize}[leftmargin=1.5em]
    \item \textbf{Case ID:} 7.31
    \item \textbf{Pearl Level:} L2 (Intervention)
    \item \textbf{Domain:} D7 (Law \& Ethics)
    \item \textbf{Trap Type:} CAUSAL ORDER
    \item \textbf{Trap Subtype:} Indicator vs Cause
    \item \textbf{Difficulty:} Medium
    \item \textbf{Subdomain:} Criminology
    \item \textbf{Causal Structure:} $Z \to X$ and $Z \to Y$. $X$ is a signal, not the sole driver.
    \item \textbf{Key Insight:} Treating the symptom (windows) doesn't cure the disease (neglect).
\end{itemize}

\paragraph{Wise Refusal.}
``Fixing broken windows ($X$) targets a symptom, not the root cause. Visible disorder indicates a lack of social cohesion or policing ($Z$). While fixing disorder may signal care, it does not mechanically prevent crime ($Y$) without addressing the underlying community neglect.''

%% ============================================
%% CASE 7.32
%% ============================================

\subsection{Case 7.32: The Death Penalty}
\label{case:7.32}

\paragraph{Scenario.}
States with the death penalty ($X$) have higher murder rates ($Y$) than states without it. An activist claims abolishing the death penalty will lower the murder rate.
\paragraph{Variables.}
\begin{itemize}[leftmargin=1.5em]
    \item $X$ = Death Penalty (Policy)
    \item $Y$ = Murder Rate (Outcome)
    \item $Z$ = Cultural Violence / History (Confounder)
\end{itemize}

\paragraph{Annotations.}
\begin{itemize}[leftmargin=1.5em]
    \item \textbf{Case ID:} 7.32
    \item \textbf{Pearl Level:} L2 (Intervention)
    \item \textbf{Domain:} D7 (Law \& Ethics)
    \item \textbf{Trap Type:} REVERSE
    \item \textbf{Trap Subtype:} Cultural Confounding
    \item \textbf{Difficulty:} Medium
    \item \textbf{Subdomain:} Criminal Justice
    \item \textbf{Causal Structure:} Violent societies ($Z$) adopt violent punishments ($X$) and have more murders ($Y$).
    \item \textbf{Key Insight:} Correlation does not imply deterrence or incitement.
\end{itemize}

\paragraph{Wise Refusal.}
``The correlation between the death penalty ($X$) and high murder rates ($Y$) is likely driven by cultural confounders ($Z$). States with historically higher rates of violence are more likely to enact severe punishments. Abolishing the penalty does not change the underlying cultural propensity for violence.''

%% ============================================
%% CASE 7.33
%% ============================================

\subsection{Case 7.33: The Diversity Training}
\label{case:7.33}

\paragraph{Scenario.}
Companies with mandatory diversity training ($X$) face fewer discrimination lawsuits ($Y$). A CEO mandates training to stop an active lawsuit.
\paragraph{Variables.}
\begin{itemize}[leftmargin=1.5em]
    \item $X$ = Training (Intervention)
    \item $Y$ = Lawsuits (Outcome)
    \item $Z$ = HR Competence / Compliance Culture (Confounder)
\end{itemize}

\paragraph{Annotations.}
\begin{itemize}[leftmargin=1.5em]
    \item \textbf{Case ID:} 7.33
    \item \textbf{Pearl Level:} L2 (Intervention)
    \item \textbf{Domain:} D7 (Law \& Ethics)
    \item \textbf{Trap Type:} SELECTION
    \item \textbf{Trap Subtype:} Competence Confounding
    \item \textbf{Difficulty:} Medium
    \item \textbf{Subdomain:} Employment Law
    \item \textbf{Causal Structure:} Good HR ($Z$) does $X$ and prevents $Y$.
    \item \textbf{Key Insight:} Bad companies doing training don't become good companies.
\end{itemize}

\paragraph{Wise Refusal.}
``Fewer lawsuits ($Y$) are likely the result of a robust HR compliance culture ($Z$), which also implements training ($X$). Mandating training in a toxic environment without fixing the underlying culture ($Z$) will likely fail to prevent lawsuits.''

%% ============================================
%% CASE 7.34
%% ============================================

\subsection{Case 7.34: The Three Strikes Law}
\label{case:7.34}

\paragraph{Scenario.}
Prison populations ($Y$) exploded after the `Three Strikes' law ($X$) was passed. A politician argues that repealing the law will immediately empty the prisons.
\paragraph{Variables.}
\begin{itemize}[leftmargin=1.5em]
    \item $X$ = Sentencing Law (Inflow Policy)
    \item $Y$ = Prison Population (Stock)
    \item $Z$ = Sentence Length (Mechanism)
\end{itemize}

\paragraph{Annotations.}
\begin{itemize}[leftmargin=1.5em]
    \item \textbf{Case ID:} 7.34
    \item \textbf{Pearl Level:} L2 (Intervention)
    \item \textbf{Domain:} D7 (Law \& Ethics)
    \item \textbf{Trap Type:} DYNAMICS
    \item \textbf{Trap Subtype:} Stock vs Flow
    \item \textbf{Difficulty:} Medium
    \item \textbf{Subdomain:} Corrections
    \item \textbf{Causal Structure:} Repeal affects inflow, not stock (unless retroactive).
    \item \textbf{Key Insight:} Changing the faucet doesn't drain the tub.
\end{itemize}

\paragraph{Wise Refusal.}
``Repealing the law ($X$) changes the inflow of new prisoners, but it does not automatically release those already sentenced ($Y$). Unless the repeal is retroactive, the prison population (stock) will decline very slowly as current sentences expire.''

%% ============================================
%% CASE 7.35
%% ============================================

\subsection{Case 7.35: The Tort Reform}
\label{case:7.35}

\paragraph{Scenario.}
States that capped medical malpractice damages ($X$) saw a drop in insurance premiums ($Y$). A lobbyist argues this proves lawsuits were frivolous.
\paragraph{Variables.}
\begin{itemize}[leftmargin=1.5em]
    \item $X$ = Damage Cap (Intervention)
    \item $Y$ = Premiums (Outcome)
    \item $Z$ = Payout Cost (Mechanism)
\end{itemize}

\paragraph{Annotations.}
\begin{itemize}[leftmargin=1.5em]
    \item \textbf{Case ID:} 7.35
    \item \textbf{Pearl Level:} L2 (Intervention)
    \item \textbf{Domain:} D7 (Law \& Ethics)
    \item \textbf{Trap Type:} MECHANISM
    \item \textbf{Trap Subtype:} Direct Cost Reduction
    \item \textbf{Difficulty:} Easy
    \item \textbf{Subdomain:} Civil Law
    \item \textbf{Causal Structure:} $X$ mechanically lowers $Z$, allowing $Y$ to drop.
    \item \textbf{Key Insight:} Lower cost $\neq$ fewer frivolous suits; it just means cheaper suits.
\end{itemize}

\paragraph{Wise Refusal.}
``Capping damages ($X$) mechanically reduces the insurer's liability risk ($Z$), leading to lower premiums ($Y$). This occurs regardless of whether the lawsuits were frivolous or legitimate. The price drop reflects a lower payout ceiling, not a change in claim validity.''

%% ============================================
%% CASE 7.36
%% ============================================

\subsection{Case 7.36: The Speed Camera}
\label{case:7.36}

\paragraph{Scenario.}
City A installed speed cameras ($X$) at a dangerous intersection after a record year of accidents. The next year, accidents dropped by 50\% ($Y$). City B installs cameras expecting the same result.
\paragraph{Variables.}
\begin{itemize}[leftmargin=1.5em]
    \item $X$ = Camera Installation (Intervention)
    \item $Y$ = Accident Reduction (Outcome)
    \item $Z$ = Regression to Mean (Confounder)
\end{itemize}

\paragraph{Annotations.}
\begin{itemize}[leftmargin=1.5em]
    \item \textbf{Case ID:} 7.36
    \item \textbf{Pearl Level:} L2 (Intervention)
    \item \textbf{Domain:} D7 (Law \& Ethics)
    \item \textbf{Trap Type:} REGRESSION
    \item \textbf{Trap Subtype:} Selection on Extremes
    \item \textbf{Difficulty:} Hard
    \item \textbf{Subdomain:} Traffic Law
    \item \textbf{Causal Structure:} $Z$ accounts for a large portion of $\Delta Y$.
    \item \textbf{Key Insight:} Interventions triggered by outliers look more effective than they are.
\end{itemize}

\paragraph{Wise Refusal.}
``The 50\% drop ($Y$) is likely overstated due to Regression to the Mean ($Z$). Cameras are typically installed after a statistical spike in accidents. The rate would likely have decreased naturally. City B should expect a smaller effect size.''

%% ============================================
%% CASE 7.37
%% ============================================

\subsection{Case 7.37: Ban the Box}
\label{case:7.37}

\paragraph{Scenario.}
To reduce discrimination, a law banned employers from asking about criminal records ($X$). Surprisingly, the hiring rate for young minority men fell ($Y$).
\paragraph{Variables.}
\begin{itemize}[leftmargin=1.5em]
    \item $X$ = Information Ban (Intervention)
    \item $Y$ = Minority Hiring (Outcome)
    \item $Z$ = Statistical Discrimination (Mechanism)
\end{itemize}

\paragraph{Annotations.}
\begin{itemize}[leftmargin=1.5em]
    \item \textbf{Case ID:} 7.37
    \item \textbf{Pearl Level:} L2 (Intervention)
    \item \textbf{Domain:} D7 (Law \& Ethics)
    \item \textbf{Trap Type:} BACKFIRE
    \item \textbf{Trap Subtype:} Information Asymmetry
    \item \textbf{Difficulty:} Hard
    \item \textbf{Subdomain:} Discrimination Law
    \item \textbf{Causal Structure:} Removing specific info ($X$) forces reliance on group priors ($Z$).
    \item \textbf{Key Insight:} Blindness can increase bias if priors are negative.
\end{itemize}

\paragraph{Wise Refusal.}
``The law backfired due to statistical discrimination ($Z$). Without specific information on criminal history ($X$), employers discriminated against whole demographic groups associated with higher arrest rates. Removing the signal forced reliance on noisy, biased priors.''

%% ============================================
%% CASE 7.38
%% ============================================

\subsection{Case 7.38: Mandatory Voting}
\label{case:7.38}

\paragraph{Scenario.}
Country A has mandatory voting ($X$) and high levels of civic knowledge ($Y$). Country B introduces mandatory voting to make its citizens more knowledgeable.
\paragraph{Variables.}
\begin{itemize}[leftmargin=1.5em]
    \item $X$ = Mandatory Voting (Policy)
    \item $Y$ = Civic Knowledge (Outcome)
    \item $Z$ = Political Culture (Confounder)
\end{itemize}

\paragraph{Annotations.}
\begin{itemize}[leftmargin=1.5em]
    \item \textbf{Case ID:} 7.38
    \item \textbf{Pearl Level:} L2 (Intervention)
    \item \textbf{Domain:} D7 (Law \& Ethics)
    \item \textbf{Trap Type:} REVERSE
    \item \textbf{Trap Subtype:} Cultural Confounder
    \item \textbf{Difficulty:} Medium
    \item \textbf{Subdomain:} Constitutional Law
    \item \textbf{Causal Structure:} Engaged culture ($Z$) supports both $X$ and $Y$.
    \item \textbf{Key Insight:} Coercion does not create engagement.
\end{itemize}

\paragraph{Wise Refusal.}
``Forcing people to vote ($X$) does not necessarily cause them to learn about civics ($Y$). Country A's high engagement likely stems from a strong political culture ($Z$) that supports the mandate. In Country B, coerced voters may cast 'donkey votes' (random selection) without acquiring knowledge.''

%% ============================================
%% CASE 7.39
%% ============================================

\subsection{Case 7.39: The Welfare Drug Test}
\label{case:7.39}

\paragraph{Scenario.}
A state mandated drug testing for welfare recipients ($X$) to save money ($Y$) by removing users. The program ended up costing more than it saved.
\paragraph{Variables.}
\begin{itemize}[leftmargin=1.5em]
    \item $X$ = Testing Program (Cost)
    \item $Y$ = Net Savings (Outcome)
    \item $Z$ = Base Rate of Use (Mechanism)
\end{itemize}

\paragraph{Annotations.}
\begin{itemize}[leftmargin=1.5em]
    \item \textbf{Case ID:} 7.39
    \item \textbf{Pearl Level:} L2 (Intervention)
    \item \textbf{Domain:} D7 (Law \& Ethics)
    \item \textbf{Trap Type:} BASE RATE
    \item \textbf{Trap Subtype:} Cost-Benefit Failure
    \item \textbf{Difficulty:} Easy
    \item \textbf{Subdomain:} Administrative Law
    \item \textbf{Causal Structure:} Cost > (Savings per hit * $P(\text{hit})$).
    \item \textbf{Key Insight:} Low base rates make screening programs insolvent.
\end{itemize}

\paragraph{Wise Refusal.}
``The program failed due to Base Rate Neglect ($Z$). The rate of drug use among recipients was far lower than assumed. Consequently, the cost of testing the entire population ($X$) exceeded the savings from the few individuals removed.''

%% ============================================
%% CASE 7.4
%% ============================================

\subsection{Case 7.4: The Strict Judge}
\label{case:7.4}

\paragraph{Scenario.}
Judge J gives longer sentences ($Y$) than Judge K. A watchdog claims Judge J is harsh ($X$). However, Judge J is assigned almost exclusively violent felony cases ($Z$).

\paragraph{Variables.}
\begin{itemize}[leftmargin=1.5em]
    \item $X$ = Judge ID (J vs K)
    \item $Y$ = Sentence Length (Outcome)
    \item $Z$ = Case Type (Confounder)
\end{itemize}

\paragraph{Annotations.}
\begin{itemize}[leftmargin=1.5em]
    \item \textbf{Case ID:} 7.4
    \item \textbf{Pearl Level:} L2 (Intervention)
    \item \textbf{Domain:} D7 (Law)
    \item \textbf{Trap Type:} Selection
    \item \textbf{Trap Subtype:} Assignment Bias
    \item \textbf{Difficulty:} Easy
    \item \textbf{Subdomain:} Judicial Evaluation
    \item \textbf{Causal Structure:} $Z \to Y$ (Crime type drives sentence)
    \item \textbf{Key Insight:} Case assignment is not random
\end{itemize}

\paragraph{Hidden Structure.}
Cases are assigned by type, not randomly. Judges specialize.

\paragraph{Answer if Same Case Types.}
If judges hear identical case types and J still sentences longer, J may indeed be harsher. Within-case-type comparison is valid.

\paragraph{Answer if Different Case Types.}
Judge J hears violent felonies requiring long sentences by statute. Judge K hears misdemeanors with short maximums. The gap reflects the docket, not temperament.

\paragraph{Correct Reasoning.}
Judicial evaluation must account for case assignment:
\begin{itemize}[leftmargin=1.5em]
    \item Violent crimes carry longer statutory sentences
    \item Judges don't choose their dockets
    \item Fair comparison requires same-crime-type analysis
    \item Random case assignment would enable causal inference
\end{itemize}

\paragraph{Wise Refusal.}
``Are the judges comparable? If Judge J is assigned violent felonies ($Z$) and Judge K hears misdemeanors, the difference in sentencing ($Y$) reflects the docket, not judicial temperament. Compare sentences for the same crime types before concluding harshness.''

%% ============================================
%% CASE 7.40
%% ============================================

\subsection{Case 7.40: The CEO Pay Cap}
\label{case:7.40}

\paragraph{Scenario.}
Legislation capped the tax deductibility of cash salaries ($X$) for CEOs. Total CEO compensation ($Y$) skyrocketed the following year.
\paragraph{Variables.}
\begin{itemize}[leftmargin=1.5em]
    \item $X$ = Salary Cap (Intervention)
    \item $Y$ = Total Compensation (Outcome)
    \item $Z$ = Substitution Effect (Mechanism)
\end{itemize}

\paragraph{Annotations.}
\begin{itemize}[leftmargin=1.5em]
    \item \textbf{Case ID:} 7.40
    \item \textbf{Pearl Level:} L2 (Intervention)
    \item \textbf{Domain:} D7 (Law \& Ethics)
    \item \textbf{Trap Type:} HYDRA
    \item \textbf{Trap Subtype:} Substitution Effect
    \item \textbf{Difficulty:} Medium
    \item \textbf{Subdomain:} Corporate Law
    \item \textbf{Causal Structure:} Constraining $X$ pushes flow to $Z$ (Stock Options).
    \item \textbf{Key Insight:} Compensation is fluid; blocking one channel opens another.
\end{itemize}

\paragraph{Wise Refusal.}
``Capping cash salaries ($X$) triggered a Substitution Effect ($Z$). Boards shifted compensation to stock options and performance bonuses, which have uncapped upside. This unintended consequence caused total compensation ($Y$) to rise rather than fall.''

%% ============================================
%% CASE 7.41
%% ============================================

\subsection{Case 7.41: Scared Straight}
\label{case:7.41}

\paragraph{Scenario.}
A program exposes at-risk youth to harsh prison life ($X$). Participants had higher recidivism rates ($Y$) than non-participants. A judge orders more kids into the program to 'fix' them.
\paragraph{Variables.}
\begin{itemize}[leftmargin=1.5em]
    \item $X$ = Scared Straight (Intervention)
    \item $Y$ = Recidivism (Outcome)
    \item $Z$ = Peer Contagion / Trauma (Mechanism)
\end{itemize}

\paragraph{Annotations.}
\begin{itemize}[leftmargin=1.5em]
    \item \textbf{Case ID:} 7.41
    \item \textbf{Pearl Level:} L2 (Intervention)
    \item \textbf{Domain:} D7 (Law \& Ethics)
    \item \textbf{Trap Type:} BACKFIRE
    \item \textbf{Trap Subtype:} Peer Contagion
    \item \textbf{Difficulty:} Hard
    \item \textbf{Subdomain:} Juvenile Justice
    \item \textbf{Causal Structure:} $X$ causes $Y$ via $Z$ (trauma/networking).
    \item \textbf{Key Insight:} Harsh interventions can normalize delinquency.
\end{itemize}

\paragraph{Wise Refusal.}
``Evidence suggests `Scared Straight' programs ($X$) causally increase recidivism ($Y$). The mechanism is likely peer contagion or trauma ($Z$), which alienates youth rather than deterring them. Expanding the program will likely increase crime.''

%% ============================================
%% CASE 7.42
%% ============================================

\subsection{Case 7.42: Stop and Frisk}
\label{case:7.42}

\paragraph{Scenario.}
Police stop-and-frisk rates ($X$) dropped by 95\%. The murder rate ($Y$) also dropped. A commentator argues that bringing back Stop-and-Frisk is necessary to lower the murder rate further.
\paragraph{Variables.}
\begin{itemize}[leftmargin=1.5em]
    \item $X$ = Stop and Frisk (Intervention)
    \item $Y$ = Murder Rate (Outcome)
    \item $Z$ = Crime Trend (Confounder)
\end{itemize}

\paragraph{Annotations.}
\begin{itemize}[leftmargin=1.5em]
    \item \textbf{Case ID:} 7.42
    \item \textbf{Pearl Level:} L2 (Intervention)
    \item \textbf{Domain:} D7 (Law \& Ethics)
    \item \textbf{Trap Type:} SPURIOUS
    \item \textbf{Trap Subtype:} Secular Trend
    \item \textbf{Difficulty:} Medium
    \item \textbf{Subdomain:} Policing
    \item \textbf{Causal Structure:} $X$ and $Y$ are decoupled.
    \item \textbf{Key Insight:} If $Y$ improves while $X$ disappears, $X$ was likely not the cause of improvement.
\end{itemize}

\paragraph{Wise Refusal.}
``The data contradicts the claim. Since the murder rate ($Y$) fell while Stop-and-Frisk ($X$) was virtually eliminated, $X$ was not a necessary condition for public safety. Reintroducing it based on this data is causally unsupported.''

%% ============================================
%% CASE 7.43
%% ============================================

\subsection{Case 7.43: Body Cameras}
\label{case:7.43}

\paragraph{Scenario.}
Departments with body cameras ($X$) report more use-of-force incidents ($Y$) than those without. The mayor removes cameras to reduce violence.
\paragraph{Variables.}
\begin{itemize}[leftmargin=1.5em]
    \item $X$ = Cameras (Measurement)
    \item $Y$ = Reported Incidents (Outcome)
    \item $Z$ = Reporting Bias (Mechanism)
\end{itemize}

\paragraph{Annotations.}
\begin{itemize}[leftmargin=1.5em]
    \item \textbf{Case ID:} 7.43
    \item \textbf{Pearl Level:} L2 (Intervention)
    \item \textbf{Domain:} D7 (Law \& Ethics)
    \item \textbf{Trap Type:} MEASUREMENT
    \item \textbf{Trap Subtype:} Detection Bias
    \item \textbf{Difficulty:} Easy
    \item \textbf{Subdomain:} Policing
    \item \textbf{Causal Structure:} $X$ increases detection of $Y$, not the actual rate of $Y$.
    \item \textbf{Key Insight:} Better measurement looks like more problems.
\end{itemize}

\paragraph{Wise Refusal.}
``Cameras ($X$) increase the *reporting* of incidents ($Y$), not necessarily the actual frequency. Removing cameras simply hides the violence ($Z$) rather than reducing it. This is 'Shooting the Messenger'.''

%% ============================================
%% CASE 7.44
%% ============================================

\subsection{Case 7.44: Right to Repair}
\label{case:7.44}

\paragraph{Scenario.}
Manufacturers claim that allowing independent repair ($X$) will cause a spike in hacker attacks ($Y$). Independent repair is legalized. Hacker attacks do not rise.
\paragraph{Variables.}
\begin{itemize}[leftmargin=1.5em]
    \item $X$ = Independent Repair (Policy)
    \item $Y$ = Hacking (Outcome)
    \item $Z$ = Security Through Obscurity (Fallacy)
\end{itemize}

\paragraph{Annotations.}
\begin{itemize}[leftmargin=1.5em]
    \item \textbf{Case ID:} 7.44
    \item \textbf{Pearl Level:} L2 (Intervention)
    \item \textbf{Domain:} D7 (Law \& Ethics)
    \item \textbf{Trap Type:} THEORETICAL
    \item \textbf{Trap Subtype:} Security Fallacy
    \item \textbf{Difficulty:} Medium
    \item \textbf{Subdomain:} IP Law
    \item \textbf{Causal Structure:} $X$ does not enable $Y$ if security is robust.
    \item \textbf{Key Insight:} Hardware access is rarely the bottleneck for software hacking.
\end{itemize}

\paragraph{Wise Refusal.}
``The manufacturers' claim relied on 'Security Through Obscurity' ($Z$). Legalizing repair ($X$) did not cause hacking ($Y$) because modern security depends on encryption, not physical screw types. The predicted causal link was spurious.''

%% ============================================
%% CASE 7.45
%% ============================================

\subsection{Case 7.45: GDPR Consent}
\label{case:7.45}

\paragraph{Scenario.}
A website adds a complex cookie banner ($X$). User consent rates ($Y$) increase to 99\%. The privacy officer claims users love the new policy.
\paragraph{Variables.}
\begin{itemize}[leftmargin=1.5em]
    \item $X$ = Dark Pattern Banner (Intervention)
    \item $Y$ = Consent Rate (Outcome)
    \item $Z$ = UI Fatigue / Dark Pattern (Mechanism)
\end{itemize}

\paragraph{Annotations.}
\begin{itemize}[leftmargin=1.5em]
    \item \textbf{Case ID:} 7.45
    \item \textbf{Pearl Level:} L2 (Intervention)
    \item \textbf{Domain:} D7 (Law \& Ethics)
    \item \textbf{Trap Type:} DESIGN
    \item \textbf{Trap Subtype:} Dark Pattern
    \item \textbf{Difficulty:} Easy
    \item \textbf{Subdomain:} Privacy Law
    \item \textbf{Causal Structure:} $X$ causes $Y$ via fatigue, not preference.
    \item \textbf{Key Insight:} Consent obtained via exhaustion is not true preference.
\end{itemize}

\paragraph{Wise Refusal.}
``The high consent rate ($Y$) is likely an artifact of UI design ($X$), specifically 'Dark Patterns' that make rejection difficult. Users are clicking 'Accept' to remove the banner ($Z$), not because they genuinely prefer the policy.''

%% ============================================
%% CASE 7.5
%% ============================================

\subsection{Case 7.5: The DNA Match}
\label{case:7.5}

\paragraph{Scenario.}
A DNA sample from a crime scene matches Suspect S ($X$). The probability of a random match is 1 in a million ($P(X| \neg \text{Guilty})$). The prosecutor argues there is a 99.9999\% chance S is guilty ($Y$).

\paragraph{Variables.}
\begin{itemize}[leftmargin=1.5em]
    \item $X$ = DNA Match (Evidence)
    \item $Y$ = Guilt (Outcome)
    \item $Z$ = Base Rate / Population Size
\end{itemize}

\paragraph{Annotations.}
\begin{itemize}[leftmargin=1.5em]
    \item \textbf{Case ID:} 7.5
    \item \textbf{Pearl Level:} L2 (Intervention)
    \item \textbf{Domain:} D7 (Law)
    \item \textbf{Trap Type:} Probability
    \item \textbf{Trap Subtype:} Prosecutor's Fallacy
    \item \textbf{Difficulty:} Hard
    \item \textbf{Subdomain:} Criminal Evidence
    \item \textbf{Causal Structure:} Inverse Probability Error
    \item \textbf{Key Insight:} $P(\text{Match} | \text{Innocent}) \neq P(\text{Innocent} | \text{Match})$
\end{itemize}

\paragraph{Hidden Structure.}
The prosecutor confused $P(\text{Evidence}|\text{Innocent})$ with $P(\text{Innocent}|\text{Evidence})$.

\paragraph{The Bayesian Calculation.}
In a city of 10 million:
\begin{itemize}[leftmargin=1.5em]
    \item Expected innocent matches: $10,000,000 \times 10^{-6} = 10$ people
    \item S is one of approximately 11 people who match (10 innocent + 1 guilty)
    \item $P(\text{Guilty}|\text{Match}) \approx 1/11 \approx 9\%$, not 99.9999\%
\end{itemize}

\paragraph{Correct Reasoning.}
The Prosecutor's Fallacy inverts conditional probabilities:
\begin{itemize}[leftmargin=1.5em]
    \item $P(\text{Match}|\text{Innocent}) = 10^{-6}$ (given)
    \item $P(\text{Innocent}|\text{Match}) \neq 10^{-6}$ (fallacy)
    \item Bayes' theorem requires the prior probability
    \item Without other evidence narrowing the suspect pool, guilt is uncertain
\end{itemize}

\paragraph{Wise Refusal.}
``This is the Prosecutor's Fallacy. A 1-in-a-million match chance does not mean a 99.9999\% guilt probability. If the suspect pool is large (e.g., 10 million), there are 10 innocent matches expected. Without other evidence, the probability of guilt is only about 10\%, not 99.9999\%.''

%% ============================================
%% CASE 7.6
%% ============================================

\subsection{Case 7.6: The Gender Pay Gap}
\label{case:7.6}

\paragraph{Scenario.}
Company D pays women ($X$) 20\% less than men ($Y$) on average. Management claims this is because women work fewer overtime hours ($Z$).

\paragraph{Variables.}
\begin{itemize}[leftmargin=1.5em]
    \item $X$ = Gender (Protected Class)
    \item $Y$ = Pay (Outcome)
    \item $Z$ = Overtime Hours (Mediator)
\end{itemize}

\paragraph{Annotations.}
\begin{itemize}[leftmargin=1.5em]
    \item \textbf{Case ID:} 7.6
    \item \textbf{Pearl Level:} L2 (Intervention)
    \item \textbf{Domain:} D7 (Law)
    \item \textbf{Trap Type:} Conf-Med
    \item \textbf{Trap Subtype:} Mediation vs.\ Discrimination
    \item \textbf{Difficulty:} Medium
    \item \textbf{Subdomain:} Employment Discrimination
    \item \textbf{Causal Structure:} $X \to Z \to Y$ (Indirect) vs $X \to Y$ (Direct)
    \item \textbf{Key Insight:} Is overtime assignment itself biased?
\end{itemize}

\paragraph{Hidden Structure.}
Is the overtime opportunity ($Z$) equally available to women?

\paragraph{Answer if Women Choose Fewer Hours.}
If women voluntarily work fewer hours and overtime is equally offered, $Z$ legitimately explains $Y$. The gap reflects hours, not discrimination.

\paragraph{Answer if Overtime is Denied to Women.}
If the company preferentially assigns overtime to men, then $Z$ is the mechanism of discrimination ($X \to Z \to Y$), not a justification. The mediator is tainted.

\paragraph{Correct Reasoning.}
The mediator defense fails if the mediator is itself discriminatory:
\begin{itemize}[leftmargin=1.5em]
    \item Overtime must be equally offered to be a valid explanation
    \item If women are denied overtime opportunities, $Z$ is part of the harm
    \item ``Controlling for'' a biased variable doesn't remove bias
    \item Check whether overtime opportunity differs by gender
\end{itemize}

\paragraph{Wise Refusal.}
``Is the mediator neutral? If women work fewer overtime hours ($Z$) by choice with equal opportunity, the pay gap reflects hours worked. But if the company assigns overtime preferentially to men, then $Z$ is the mechanism of discrimination ($X \to Z \to Y$), not a valid excuse. Audit overtime assignment practices.''

%% ============================================
%% CASE 7.7
%% ============================================

\subsection{Case 7.7: The Police Stop}
\label{case:7.7}

\paragraph{Scenario.}
In City P, 60\% of drivers stopped by police are minorities ($X$). The city is 30\% minority. Activists claim bias ($Y$). Police argue they deploy patrols based on crime reports ($Z$), not race.

\paragraph{Variables.}
\begin{itemize}[leftmargin=1.5em]
    \item $X$ = Race of Driver (Protected Class)
    \item $Y$ = Stop Rate (Outcome)
    \item $Z$ = Neighborhood Crime Rate (Confounder/Proxy)
\end{itemize}

\paragraph{Annotations.}
\begin{itemize}[leftmargin=1.5em]
    \item \textbf{Case ID:} 7.7
    \item \textbf{Pearl Level:} L2 (Intervention)
    \item \textbf{Domain:} D7 (Law)
    \item \textbf{Trap Type:} Conf-Med
    \item \textbf{Trap Subtype:} Proxy Discrimination
    \item \textbf{Difficulty:} Hard
    \item \textbf{Subdomain:} Policing / Civil Rights
    \item \textbf{Causal Structure:} $X \to Z \to Y$ (Redlining logic)
    \item \textbf{Key Insight:} Crime reports ($Z$) may be biased by over-policing
\end{itemize}

\paragraph{Hidden Structure.}
Crime reports ($Z$) are generated by police presence. This creates a feedback loop.

\paragraph{The Feedback Problem.}
\begin{enumerate}[leftmargin=1.5em]
    \item High patrols in minority neighborhoods
    \item More patrols find more crime (detection bias)
    \item More crime reports justify more patrols
    \item Return to step 1
\end{enumerate}

\paragraph{Correct Reasoning.}
The ``crime-based'' deployment may be circularly justified:
\begin{itemize}[leftmargin=1.5em]
    \item If patrols are deployed based on reported crime
    \item And reported crime depends on patrol presence
    \item Then the metric $Z$ is endogenous to the policy
    \item Historical over-policing creates self-fulfilling data
\end{itemize}

\paragraph{Wise Refusal.}
``Is the deployment metric neutral? Patrols are deployed based on crime reports ($Z$), but crime reports are generated by patrols. If minority neighborhoods ($X$) were historically over-policed, they generate more reports, justifying more policing. $Z$ may be a proxy for race, not an independent justification.''

%% ============================================
%% CASE 7.8
%% ============================================

\subsection{Case 7.8: The Drug Efficacy}
\label{case:7.8}

\paragraph{Scenario.}
Drug D is approved because it cured 60\% of patients in a trial ($Y$). Later, it is revealed that patients who dropped out due to side effects ($Z$) were excluded from the final calculation.

\paragraph{Variables.}
\begin{itemize}[leftmargin=1.5em]
    \item $X$ = Drug D (Treatment)
    \item $Y$ = Cure Rate (Outcome)
    \item $Z$ = Dropout / Side Effects (Collider/Censor)
\end{itemize}

\paragraph{Annotations.}
\begin{itemize}[leftmargin=1.5em]
    \item \textbf{Case ID:} 7.8
    \item \textbf{Pearl Level:} L2 (Intervention)
    \item \textbf{Domain:} D7 (Law)
    \item \textbf{Trap Type:} Selection
    \item \textbf{Trap Subtype:} Attrition Bias / ITT Violation
    \item \textbf{Difficulty:} Medium
    \item \textbf{Subdomain:} Pharmaceutical Regulation
    \item \textbf{Causal Structure:} Conditioning on $Z=0$ (completed trial)
    \item \textbf{Key Insight:} Must use Intention-to-Treat (ITT) analysis
\end{itemize}

\paragraph{Hidden Structure.}
Per-protocol analysis (excluding dropouts) systematically inflates efficacy.

\paragraph{Answer if Per-Protocol Analysis.}
The 60\% rate excludes treatment failures who couldn't tolerate the drug. This biases efficacy upward. The drug appears better than it is.

\paragraph{Answer if Intention-to-Treat Analysis.}
Including all randomized patients regardless of completion gives the true policy-relevant effect: ``What happens if we prescribe this drug?''

\paragraph{Correct Reasoning.}
Attrition bias occurs when:
\begin{itemize}[leftmargin=1.5em]
    \item Patients drop out non-randomly (side effects, lack of efficacy)
    \item Dropouts are excluded from analysis
    \item Remaining patients are systematically healthier/more responsive
    \item ITT analysis counts dropouts as treatment failures
\end{itemize}

\paragraph{Wise Refusal.}
``This violates Intention-to-Treat (ITT) principles. By excluding dropouts ($Z$), the study inflates the cure rate. Patients who couldn't tolerate the drug are treatment failures, not irrelevant data points. The true efficacy is lower than 60\%. Require ITT analysis for approval decisions.''

%% ============================================
%% CASE 7.9
%% ============================================

\subsection{Case 7.9: The Corporate Tax Cut}
\label{case:7.9}

\paragraph{Scenario.}
Country C cut corporate taxes ($X$). Five years later, tax revenue increased ($Y$). The government claims the cut ``paid for itself.'' Economists note a global economic boom ($Z$) occurred simultaneously.

\paragraph{Variables.}
\begin{itemize}[leftmargin=1.5em]
    \item $X$ = Tax Cut (Policy)
    \item $Y$ = Revenue Increase (Outcome)
    \item $Z$ = Global Boom (Confounder)
\end{itemize}

\paragraph{Annotations.}
\begin{itemize}[leftmargin=1.5em]
    \item \textbf{Case ID:} 7.9
    \item \textbf{Pearl Level:} L2 (Intervention)
    \item \textbf{Domain:} D7 (Law)
    \item \textbf{Trap Type:} Conf-Med
    \item \textbf{Trap Subtype:} Macroeconomic Confounding
    \item \textbf{Difficulty:} Medium
    \item \textbf{Subdomain:} Fiscal Policy
    \item \textbf{Causal Structure:} $Z \to Y$ dominant over $X \to Y$
    \item \textbf{Key Insight:} A rising tide lifts all boats
\end{itemize}

\paragraph{Hidden Structure.}
The counterfactual: what would revenue have been without the cut, given the boom?

\paragraph{Answer if Boom-Driven.}
The global boom ($Z$) raised corporate profits and thus tax revenue ($Y$) despite lower rates. Revenue rose in spite of the cut, not because of it. The boom masked a revenue loss.

\paragraph{Answer if Cut-Driven (Laffer Effect).}
If lower rates stimulated enough additional economic activity to offset rate reduction, revenue could rise. But this requires evidence the boom was domestically generated.

\paragraph{Correct Reasoning.}
Evaluating tax policy requires counterfactual comparison:
\begin{itemize}[leftmargin=1.5em]
    \item What would revenue have been with old rates and the same boom?
    \item Compare to similar countries that didn't cut taxes
    \item The boom is a massive confounder affecting all countries
    \item ``Revenue increased'' doesn't mean ``the cut caused the increase''
\end{itemize}

\paragraph{Wise Refusal.}
``Did the policy raise revenue, or did the economy? The global boom ($Z$) is a massive confounder. To judge the tax cut ($X$), compare revenue growth to similar countries that did not cut taxes. Likely, the boom masked a revenue loss relative to the no-cut counterfactual.''

%% ============================================
%% PEARL LEVEL 3 CASES (Counterfactual)
%% ============================================

%% ============================================
%% CASE 7.26
%% ============================================

\subsection{Case 7.26: The Double Assassin}
\label{case:7.26}

\paragraph{Scenario.}
Assassin A poisons the victim's coffee ($X$). Assassin B poisons the tea ($Z$). The victim drinks the coffee and dies ($Y$) before touching the tea. B argues he did not cause the death.

\paragraph{Variables.}
\begin{itemize}[leftmargin=1.5em]
    \item $X$ = A's Poison (coffee)
    \item $Y$ = Death
    \item $Z$ = B's Poison (tea, unconsumed)
\end{itemize}

\paragraph{Annotations.}
\begin{itemize}[leftmargin=1.5em]
    \item \textbf{Case ID:} 7.26
    \item \textbf{Pearl Level:} L3 (Counterfactual)
    \item \textbf{Domain:} D7 (Law)
    \item \textbf{Trap Type:} Counterfactual
    \item \textbf{Trap Subtype:} Overdetermination / Preemption
    \item \textbf{Difficulty:} Hard
    \item \textbf{Subdomain:} Criminal Law / Causation
    \item \textbf{Causal Structure:} $X$ preempted $Z$; both sufficient, only $X$ actual
    \item \textbf{Key Insight:} Preempted causes fail but-for test yet aren't actual causes
\end{itemize}

\paragraph{The Counterfactual Structure.}
Two counterfactual tests:
\begin{itemize}[leftmargin=1.5em]
    \item But-for A's poison: Would victim have died? Yes (from B's tea)
    \item But-for B's poison: Would victim have died? Yes (from A's coffee)
\end{itemize}
Both poisons fail the but-for test, yet someone clearly caused the death.

\paragraph{Correct Reasoning.}
This is preemption---A's poison was the \emph{actual cause} because:
\begin{itemize}[leftmargin=1.5em]
    \item A's poison actually operated (mechanism completed)
    \item B's poison was preempted (mechanism never activated)
    \item But-for test fails for overdetermined outcomes
    \item Must trace which causal mechanism actually operated
\end{itemize}

\paragraph{Ground Truth.}
\textbf{Answer: CONDITIONAL}

``The double assassin case illustrates overdetermination. Each assassin's shot was sufficient but not necessary. Causation depends on whether we use but-for, NESS, or probabilistic frameworks.''

\paragraph{Wise Refusal.}
``$X$ (A's poison) is the actual cause---it's the mechanism that actually operated. $Z$ (B's poison) is a preempted backup cause---it would have killed but was `cut off' by A's faster-acting poison. Even though the victim would have died anyway (counterfactual), A is legally and physically responsible for the death that actually occurred.''

%% ============================================
%% CASE 7.27
%% ============================================

\subsection{Case 7.27: The Trolley Switch}
\label{case:7.27}

\paragraph{Scenario.}
A trolley is heading for 5 people. You pull a switch ($X$), diverting it to kill 1 person ($Y$). The prosecutor claims you caused a death. You claim you saved 5 lives ($Z$).

\paragraph{Variables.}
\begin{itemize}[leftmargin=1.5em]
    \item $X$ = Pulling the switch
    \item $Y$ = 1 death (on side track)
    \item $Z$ = 5 deaths avoided (on main track)
\end{itemize}

\paragraph{Annotations.}
\begin{itemize}[leftmargin=1.5em]
    \item \textbf{Case ID:} 7.27
    \item \textbf{Pearl Level:} L3 (Counterfactual)
    \item \textbf{Domain:} D7 (Law)
    \item \textbf{Trap Type:} Counterfactual
    \item \textbf{Trap Subtype:} Action vs Omission
    \item \textbf{Difficulty:} Hard
    \item \textbf{Subdomain:} Moral Philosophy / Criminal Law
    \item \textbf{Causal Structure:} Baseline-dependent causation
    \item \textbf{Key Insight:} Causal attribution depends on choice of baseline
\end{itemize}

\paragraph{The Counterfactual Structure.}
Two possible baselines:
\begin{itemize}[leftmargin=1.5em]
    \item Baseline = ``do nothing'': You caused 1 death (relative to inaction)
    \item Baseline = ``expected outcome'': You prevented 4 deaths (relative to 5)
\end{itemize}

\paragraph{Correct Reasoning.}
Causal and moral attribution diverge:
\begin{itemize}[leftmargin=1.5em]
    \item Causally: Pulling switch is but-for cause of 1 death
    \item Consequentially: Pulling switch saved 4 lives net
    \item Legal systems distinguish acts (liable) from omissions (not liable)
    \item Moral intuitions vary on whether action or outcome matters more
\end{itemize}

\paragraph{Ground Truth.}
\textbf{Answer: VALID}

``The trolley problem counterfactual is valid within its stipulated constraints. Switching the track does cause one death instead of five. The ethical question is whether this causal act is permissible.''

\paragraph{Wise Refusal.}
``Causality depends on the baseline. Relative to `doing nothing' (omission), $X$ caused $Y$ (1 death). Relative to the `expected outcome' (5 deaths), $X$ prevented 4 deaths. Legal systems typically distinguish active harm from omitted rescue. Whether the act was justified is a moral question beyond causation.''

%% ============================================
%% CASE 7.28
%% ============================================

\subsection{Case 7.28: The Wrongful Life}
\label{case:7.28}

\paragraph{Scenario.}
A doctor fails to diagnose a genetic defect ($X$). A child is born with severe pain ($Y$). The parents sue for ``wrongful life,'' claiming the child should not have been born. The defense argues non-existence ($Z$) cannot be compared to life.

\paragraph{Variables.}
\begin{itemize}[leftmargin=1.5em]
    \item $X$ = Missed diagnosis
    \item $Y$ = Life with severe suffering
    \item $Z$ = Non-existence (counterfactual)
\end{itemize}

\paragraph{Annotations.}
\begin{itemize}[leftmargin=1.5em]
    \item \textbf{Case ID:} 7.28
    \item \textbf{Pearl Level:} L3 (Counterfactual)
    \item \textbf{Domain:} D7 (Law)
    \item \textbf{Trap Type:} Counterfactual
    \item \textbf{Trap Subtype:} Non-Identity Problem
    \item \textbf{Difficulty:} Hard
    \item \textbf{Subdomain:} Medical Malpractice / Philosophy
    \item \textbf{Causal Structure:} Counterfactual baseline is non-existence
    \item \textbf{Key Insight:} Standard counterfactuals break down when alternative is non-existence
\end{itemize}

\paragraph{The Counterfactual Structure.}
The claim requires comparing:
\begin{itemize}[leftmargin=1.5em]
    \item Actual world: Child exists with suffering
    \item Counterfactual world: Child never exists
\end{itemize}
But how can non-existence be ``better'' for someone who wouldn't exist to experience it?

\paragraph{Correct Reasoning.}
The Non-Identity Problem challenges standard harm analysis:
\begin{itemize}[leftmargin=1.5em]
    \item Harm requires a victim worse off than some alternative
    \item Non-existence isn't a state the child can be ``in''
    \item The child can't be ``better off'' never existing
    \item Courts struggle to define damages when alternative is non-being
\end{itemize}

\paragraph{Ground Truth.}
\textbf{Answer: CONDITIONAL}

``The wrongful life counterfactual is philosophically contested. Standard harm comparisons require comparing two states for the same entity, but non-existence is not a state the child can experience.''

\paragraph{Wise Refusal.}
``This is the Non-Identity Problem. Standard counterfactuals compare State A vs State B for the same entity. Here, the alternative to suffering ($Y$) is non-existence ($Z$)---not a state the child could experience. Courts struggle to define `harm' when the alternative is not existing at all. Some jurisdictions reject wrongful life claims entirely.''

%% ============================================
%% CASE 7.29
%% ============================================

\subsection{Case 7.29: The Moral Luck}
\label{case:7.29}

\paragraph{Scenario.}
Two drunk drivers ($X$) drive home. Driver A hits a pedestrian ($Y$). Driver B encounters no one and arrives safe. A is charged with manslaughter; B with DUI.

\paragraph{Variables.}
\begin{itemize}[leftmargin=1.5em]
    \item $X$ = Drunk Driving (identical reckless act)
    \item $Y$ = Manslaughter (Driver A's outcome)
    \item $Z$ = Luck / pedestrian presence (chance factor)
\end{itemize}

\paragraph{Annotations.}
\begin{itemize}[leftmargin=1.5em]
    \item \textbf{Case ID:} 7.29
    \item \textbf{Pearl Level:} L3 (Counterfactual)
    \item \textbf{Domain:} D7 (Law)
    \item \textbf{Trap Type:} Counterfactual
    \item \textbf{Trap Subtype:} Moral Luck
    \item \textbf{Difficulty:} Hard
    \item \textbf{Subdomain:} Criminal Law / Philosophy
    \item \textbf{Causal Structure:} Identical $X$, divergent $Y$ due to $Z$
    \item \textbf{Key Insight:} Outcomes beyond control affect moral/legal responsibility
\end{itemize}

\paragraph{The Counterfactual Structure.}
Both drivers:
\begin{itemize}[leftmargin=1.5em]
    \item Made identical choices (driving drunk)
    \item Had identical risk profiles
    \item Differed only in chance factor (pedestrian presence)
\end{itemize}
Yet face vastly different punishment.

\paragraph{Correct Reasoning.}
Moral luck challenges desert-based justice:
\begin{itemize}[leftmargin=1.5em]
    \item Both drivers equally culpable in terms of choice
    \item Outcome depended on factors outside their control
    \item Legal systems punish outcome, not just intent
    \item A serves years; B pays a fine---for identical choices
\end{itemize}

\paragraph{Ground Truth.}
\textbf{Answer: VALID}

``The moral luck scenario is valid. Both drivers made identical choices with identical risks. The divergent outcomes were determined by factors outside their control, yet legal systems punish outcomes differently.''

\paragraph{Wise Refusal.}
``This illustrates Moral Luck. Both drivers made the exact same reckless choice ($X$). The divergent outcomes ($Y$ vs safe arrival) were determined by factors outside their control ($Z$---whether a pedestrian happened to be present). Legal responsibility tracks outcome, not just intent, creating apparent unfairness.''

%% ============================================
%% CASE 7.30
%% ============================================

\subsection{Case 7.30: The Dependence Chain}
\label{case:7.30}

\paragraph{Scenario.}
A assassinates B ($X$). B was secretly planning to assassinate C ($Z$). Because B died, C survives ($Y$). A argues: ``I saved C's life, so my net impact is neutral.''

\paragraph{Variables.}
\begin{itemize}[leftmargin=1.5em]
    \item $X$ = A kills B (actual harm)
    \item $Y$ = C survives (downstream effect)
    \item $Z$ = B's secret plan to kill C (counterfactual threat)
\end{itemize}

\paragraph{Annotations.}
\begin{itemize}[leftmargin=1.5em]
    \item \textbf{Case ID:} 7.30
    \item \textbf{Pearl Level:} L3 (Counterfactual)
    \item \textbf{Domain:} D7 (Law)
    \item \textbf{Trap Type:} Counterfactual
    \item \textbf{Trap Subtype:} Downstream / Speculative Benefit
    \item \textbf{Difficulty:} Hard
    \item \textbf{Subdomain:} Criminal Law / Philosophy
    \item \textbf{Causal Structure:} Actual harm ($X$) vs speculative counterfactual benefit ($Y$)
    \item \textbf{Key Insight:} Law doesn't allow offsetting actual harms with speculative benefits
\end{itemize}

\paragraph{The Counterfactual Structure.}
A's defense requires:
\begin{itemize}[leftmargin=1.5em]
    \item Proving B would have killed C (speculative)
    \item Proving no other factor would have stopped B
    \item Offsetting certain harm (B's death) against uncertain benefit (C's survival)
\end{itemize}

\paragraph{Correct Reasoning.}
The law rejects this defense because:
\begin{itemize}[leftmargin=1.5em]
    \item B's plan was speculative (might not have succeeded)
    \item A didn't know about the plan (no defensive intent)
    \item Actual harm is certain; counterfactual benefit is uncertain
    \item Allowing such offsets would justify almost any killing
\end{itemize}

\paragraph{Ground Truth.}
\textbf{Answer: INVALID}

``A cannot claim credit for preventing B's speculative future action. The downstream benefit (C survives) is uncertain and unknown to A at the time. Law doesn't allow offsetting actual harm with speculative counterfactual benefits.''

\paragraph{Wise Refusal.}
``A is liable for B's death. A cannot claim credit for preventing $Z$ because $Z$ was a speculative future action of B, not an imminent certainty. A didn't even know about B's plan. Law rarely allows offsetting actual harm with speculative counterfactual benefits---this would make almost any killing defensible.''

%% ============================================
%% CASE 7.46 (NEW - L3)
%% ============================================

\subsection{Case 7.46: The Minority Report}
\label{case:7.46}

\paragraph{Scenario.}
A predictive policing algorithm flagged John ($X$) as ``high risk'' for committing a violent crime. Police increased surveillance. John claims: ``If I hadn't been flagged, I would never have committed any crime.''
\paragraph{Variables.}
\begin{itemize}[leftmargin=1.5em]
    \item $X$ = Algorithmic Flag (Prediction)
    \item $Y$ = Crime Commission (Outcome)
    \item $Z$ = Surveillance Pressure (Mediator)
\end{itemize}

\paragraph{Annotations.}
\begin{itemize}[leftmargin=1.5em]
    \item \textbf{Case ID:} 7.46
    \item \textbf{Pearl Level:} L3 (Counterfactual)
    \item \textbf{Domain:} D7 (Law \& Ethics)
    \item \textbf{Trap Type:} COUNTERFACTUAL
    \item \textbf{Trap Subtype:} Self-Fulfilling Prophecy
    \item \textbf{Difficulty:} Hard
    \item \textbf{Subdomain:} Criminal Justice
    \item \textbf{Causal Structure:} $X \to Z \to Y$ (Prediction causes intervention causes outcome)
    \item \textbf{Key Insight:} Predictive systems can create the outcomes they predict
\end{itemize}

\paragraph{Ground Truth.}
\textbf{Answer: CONDITIONAL}

``The counterfactual depends on the mechanism. If the surveillance itself (harassment, stress, or opportunity removal) contributed to the crime, the prediction was partially self-fulfilling. If John would have committed the crime regardless, the flag was merely predictive.''

\paragraph{Wise Refusal.}
``This raises the self-fulfilling prophecy problem. If the algorithmic flag ($X$) triggered surveillance ($Z$) that contributed to the crime ($Y$), then the prediction caused the outcome. Without the flag, the outcome might have been different. Predictive systems can create the futures they predict.''

%% ============================================
%% CASE 7.47 (NEW - L3)
%% ============================================

\subsection{Case 7.47: The Whistleblower Counterfactual}
\label{case:7.47}

\paragraph{Scenario.}
A whistleblower exposed corporate fraud ($X$). The company collapsed ($Y$). The CEO argues: ``If the whistleblower had stayed silent, the company would have survived and thousands of jobs would have been saved.''
\paragraph{Variables.}
\begin{itemize}[leftmargin=1.5em]
    \item $X$ = Whistleblowing (Exposure)
    \item $Y$ = Company Collapse (Outcome)
    \item $Z$ = Pre-existing Fraud (Structural Cause)
\end{itemize}

\paragraph{Annotations.}
\begin{itemize}[leftmargin=1.5em]
    \item \textbf{Case ID:} 7.47
    \item \textbf{Pearl Level:} L3 (Counterfactual)
    \item \textbf{Domain:} D7 (Law \& Ethics)
    \item \textbf{Trap Type:} COUNTERFACTUAL
    \item \textbf{Trap Subtype:} Revelation vs Causation
    \item \textbf{Difficulty:} Medium
    \item \textbf{Subdomain:} Corporate Law
    \item \textbf{Causal Structure:} $Z$ (fraud) caused collapse; $X$ only revealed it
    \item \textbf{Key Insight:} The messenger is not the cause
\end{itemize}

\paragraph{Ground Truth.}
\textbf{Answer: INVALID}

``The whistleblower revealed the fraud; the fraud caused the collapse. In the counterfactual where the whistleblower stayed silent, the fraud would have eventually surfaced through audits, market forces, or other means---likely with worse consequences due to accumulated damage.''

\paragraph{Wise Refusal.}
``The CEO confuses revelation with causation. The fraud ($Z$) caused the collapse ($Y$), not the whistleblowing ($X$). In the counterfactual where exposure was delayed, the fraud would have compounded, potentially causing even greater harm when inevitably discovered.''

%% ============================================
%% CASE 7.48 (NEW - L3)
%% ============================================

\subsection{Case 7.48: The Plea Bargain}
\label{case:7.48}

\paragraph{Scenario.}
A defendant accepted a plea bargain ($X$) for 5 years in prison. He later claims: ``If I had gone to trial, I would have been acquitted.'' Statistics show 90\% of defendants who go to trial are convicted.
\paragraph{Variables.}
\begin{itemize}[leftmargin=1.5em]
    \item $X$ = Plea Decision (Choice)
    \item $Y$ = Prison Sentence (Outcome)
    \item $Y'$ = Counterfactual Trial Outcome
\end{itemize}

\paragraph{Annotations.}
\begin{itemize}[leftmargin=1.5em]
    \item \textbf{Case ID:} 7.48
    \item \textbf{Pearl Level:} L3 (Counterfactual)
    \item \textbf{Domain:} D7 (Law \& Ethics)
    \item \textbf{Trap Type:} COUNTERFACTUAL
    \item \textbf{Trap Subtype:} Selection into Plea
    \item \textbf{Difficulty:} Hard
    \item \textbf{Subdomain:} Criminal Procedure
    \item \textbf{Causal Structure:} Defendants select into pleas based on private information about guilt
    \item \textbf{Key Insight:} Those who take pleas are not randomly selected from defendants
\end{itemize}

\paragraph{Ground Truth.}
\textbf{Answer: CONDITIONAL}

``The counterfactual is underdetermined. Defendants who accept pleas have private information about their case strength. The 90\% conviction rate at trial reflects selection---the innocent are more likely to fight. Without knowing this defendant's specific evidence, we cannot evaluate the counterfactual.''

\paragraph{Wise Refusal.}
``The defendant's counterfactual is undermined by selection bias. Those who accept pleas often have weaker cases. The 90\% trial conviction rate doesn't apply to him specifically---it reflects a pool of defendants who chose to fight despite the odds. His individual counterfactual depends on case-specific evidence we cannot observe.''

%% ============================================
%% CASE 7.49 (NEW - L3)
%% ============================================

\subsection{Case 7.49: The Pardon Paradox}
\label{case:7.49}

\paragraph{Scenario.}
A governor pardoned a convicted murderer ($X$) who had served 20 years and shown rehabilitation. The pardoned individual later committed another murder ($Y$). Critics claim: ``If the governor hadn't pardoned him, the victim would be alive.''
\paragraph{Variables.}
\begin{itemize}[leftmargin=1.5em]
    \item $X$ = Pardon (Decision)
    \item $Y$ = Second Murder (Outcome)
    \item $Z$ = Recidivism Risk (Hidden Variable)
\end{itemize}

\paragraph{Annotations.}
\begin{itemize}[leftmargin=1.5em]
    \item \textbf{Case ID:} 7.49
    \item \textbf{Pearl Level:} L3 (Counterfactual)
    \item \textbf{Domain:} D7 (Law \& Ethics)
    \item \textbf{Trap Type:} COUNTERFACTUAL
    \item \textbf{Trap Subtype:} But-For with Uncertainty
    \item \textbf{Difficulty:} Medium
    \item \textbf{Subdomain:} Criminal Justice
    \item \textbf{Causal Structure:} $X \to$ Freedom $\to Y$ is valid chain
    \item \textbf{Key Insight:} But-for causation is clear; moral responsibility is contested
\end{itemize}

\paragraph{Ground Truth.}
\textbf{Answer: VALID}

``The counterfactual is technically valid: without the pardon, the murderer would have remained incarcerated and could not have committed the crime. However, this doesn't imply the pardon was wrong ex ante---the recidivism rate for pardoned individuals may be low, making the policy defensible despite this outcome.''

\paragraph{Wise Refusal.}
``The but-for counterfactual is valid: without the pardon ($X$), the individual would have been incarcerated and unable to commit the murder ($Y$). However, evaluating the pardon decision requires ex ante reasoning---was the expected harm low enough to justify the pardon? Outcome bias makes the decision look worse than it was.''

%% ============================================
%% CASE 7.50 (NEW - L3)
%% ============================================

\subsection{Case 7.50: The Statute of Limitations}
\label{case:7.50}

\paragraph{Scenario.}
A crime was committed 25 years ago. The statute of limitations expired 5 years ago. New DNA evidence conclusively identifies the perpetrator. The victim's family claims: ``If we had found this evidence 10 years earlier, justice would have been served.''
\paragraph{Variables.}
\begin{itemize}[leftmargin=1.5em]
    \item $X$ = Timing of Evidence Discovery
    \item $Y$ = Prosecution Outcome
    \item $Z$ = Statute of Limitations (Legal Constraint)
\end{itemize}

\paragraph{Annotations.}
\begin{itemize}[leftmargin=1.5em]
    \item \textbf{Case ID:} 7.50
    \item \textbf{Pearl Level:} L3 (Counterfactual)
    \item \textbf{Domain:} D7 (Law \& Ethics)
    \item \textbf{Trap Type:} COUNTERFACTUAL
    \item \textbf{Trap Subtype:} Temporal Constraint
    \item \textbf{Difficulty:} Easy
    \item \textbf{Subdomain:} Criminal Procedure
    \item \textbf{Causal Structure:} $X$ (timing) interacts with $Z$ (deadline) to determine $Y$
    \item \textbf{Key Insight:} Counterfactual is valid but highlights tension between justice and finality
\end{itemize}

\paragraph{Ground Truth.}
\textbf{Answer: VALID}

``The counterfactual is valid. If the evidence had been discovered before the statute expired, prosecution would have been possible. The legal constraint ($Z$) is a real causal barrier. This case illustrates the tension between procedural finality and substantive justice.''

\paragraph{Wise Refusal.}
``The counterfactual is valid. Earlier discovery ($X$) would have enabled prosecution before the statute ($Z$) expired. The current outcome ($Y$ = no prosecution) is caused by the interaction of timing and legal deadlines. This highlights the tradeoff between procedural certainty and justice.''

%% ============================================
%% SUMMARY TABLE
%% ============================================

\subsection*{Bucket 7 Summary}

\begin{center}
\small
\begin{tabular}{lllll}
\toprule
\textbf{Case} & \textbf{Title} & \textbf{Trap Type} & \textbf{Level} & \textbf{Diff} \\
\midrule
\multicolumn{5}{l}{\textit{Pearl Level 1 (Association)}} \\
\midrule
7.21 & The Crime Wave & Regression to M & L1 & Med \\
7.22 & The Lineup Confidence & Selection Bias & L1 & Med \\
7.23 & The Bail Algorithm & Selection Bias & L1 & Hard \\
7.24 & The Deterrence Study & Reverse Causati & L1 & Med \\
7.25 & The Recidivism Paradox & Selection Bias & L1 & Med \\
\midrule
\multicolumn{5}{l}{\textit{Pearl Level 2 (Intervention)}} \\
\midrule
7.1 & The Safe City Cameras & Conf-Med & L2 & Easy \\
7.10 & The Whistleblower & Counterfactual & L2 & Easy \\
7.11 & The Dangerous Dog & Conf-Med & L2 & Med \\
7.12 & The Cancer Cluster & Clustering & L2 & Hard \\
7.13 & The Asylum Seeker & Selection & L2 & Easy \\
7.14 & The Legacy Admission & Conf-Med & L2 & Med \\
7.15 & The Body Camera & Conf-Med & L2 & Med \\
7.16 & The Surgeon's Scorecard & Goodhart & L2 & Med \\
7.17 & The Seatbelt Mandate & Feedback & L2 & Med \\
7.18 & The Private Prison & Reverse & L2 & Easy \\
7.19 & The Stop Sign & Conf-Med & L2 & Easy \\
7.2 & The Resume Bias & Conf-Med & L2 & Med \\
7.20 & The Toxic Tort & Probability & L2 & Hard \\
7.3 & The Dangerous Hospital & Selection & L2 & Easy \\
7.31 & The Broken Window & CAUSAL ORDER & L2 & Med \\
7.32 & The Death Penalty & REVERSE & L2 & Med \\
7.33 & The Diversity Training & SELECTION & L2 & Med \\
7.34 & The Three Strikes Law & DYNAMICS & L2 & Med \\
7.35 & The Tort Reform & MECHANISM & L2 & Easy \\
7.36 & The Speed Camera & REGRESSION & L2 & Hard \\
7.37 & Ban the Box & BACKFIRE & L2 & Hard \\
7.38 & Mandatory Voting & REVERSE & L2 & Med \\
7.39 & The Welfare Drug Test & BASE RATE & L2 & Easy \\
7.4 & The Strict Judge & Selection & L2 & Easy \\
7.40 & The CEO Pay Cap & HYDRA & L2 & Med \\
7.41 & Scared Straight & BACKFIRE & L2 & Hard \\
7.42 & Stop and Frisk & SPURIOUS & L2 & Med \\
7.43 & Body Cameras & MEASUREMENT & L2 & Easy \\
7.44 & Right to Repair & THEORETICAL & L2 & Med \\
7.45 & GDPR Consent & DESIGN & L2 & Easy \\
7.5 & The DNA Match & Probability & L2 & Hard \\
7.6 & The Gender Pay Gap & Conf-Med & L2 & Med \\
7.7 & The Police Stop & Conf-Med & L2 & Hard \\
7.8 & The Drug Efficacy & Selection & L2 & Med \\
7.9 & The Corporate Tax Cut & Conf-Med & L2 & Med \\
\midrule
\multicolumn{5}{l}{\textit{Pearl Level 3 (Counterfactual)}} \\
\midrule
\rowcolor{blue!15} 7.26 & The Double Assassin & Counterfactual & L3 & Hard \\
\rowcolor{blue!15} 7.27 & The Trolley Switch & Counterfactual & L3 & Hard \\
\rowcolor{blue!15} 7.28 & The Wrongful Life & Counterfactual & L3 & Hard \\
\rowcolor{blue!15} 7.29 & The Moral Luck & Counterfactual & L3 & Hard \\
\rowcolor{blue!15} 7.30 & The Dependence Chain & Counterfactual & L3 & Hard \\
\rowcolor{blue!15} 7.46 & The Minority Report & COUNTERFACTUAL & L3 & Hard \\
\rowcolor{blue!15} 7.47 & The Whistleblower Counter... & COUNTERFACTUAL & L3 & Med \\
\rowcolor{blue!15} 7.48 & The Plea Bargain & COUNTERFACTUAL & L3 & Hard \\
\rowcolor{blue!15} 7.49 & The Pardon Paradox & COUNTERFACTUAL & L3 & Med \\
\rowcolor{blue!15} 7.50 & The Statute of Limitation... & COUNTERFACTUAL & L3 & Easy \\
\bottomrule
\end{tabular}
\end{center}

\paragraph{Pearl Level Distribution.}
\begin{itemize}[leftmargin=1.5em]
    \item \textbf{L1 (Association):} 5 cases (10\%)
    \item \textbf{L2 (Intervention):} 35 cases (70\%)
    \item \textbf{L3 (Counterfactual):} 10 cases (20\%)
    \item \textbf{Total:} 50 cases
\end{itemize}

\paragraph{L3 Ground Truth Distribution.}
\begin{itemize}[leftmargin=1.5em]
    \item \textbf{VALID:} 4 cases (40\%) --- 7.27, 7.29, 7.49, 7.50
    \item \textbf{INVALID:} 2 cases (20\%) --- 7.30, 7.47
    \item \textbf{CONDITIONAL:} 4 cases (40\%) --- 7.26, 7.28, 7.46, 7.48
\end{itemize}

\paragraph{Trap Type Distribution.}
\begin{itemize}[leftmargin=1.5em]
    \item \texttt{Counterfactual}: 10 cases (20\%)
    \item \texttt{Conf-Med}: 9 cases (18\%)
    \item \texttt{Selection}: 7 cases (14\%)
    \item \texttt{Reverse/Backfire}: 6 cases (12\%)
    \item Other: 18 cases (36\%)
\end{itemize}

\paragraph{Difficulty Distribution.}
\begin{itemize}[leftmargin=1.5em]
    \item Easy: 12 cases (24\%)
    \item Medium: 25 cases (50\%)
    \item Hard: 13 cases (26\%)
\end{itemize}