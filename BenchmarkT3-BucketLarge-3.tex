%% ============================================
%% BUCKET 3: MARKETS & WEATHER
%% T³ Benchmark Standard Format (Revised & Sorted)
%% Theme: Lag Structures, Rational Expectations, Algo-Feedback
%% Total Cases: 45 (L1: 5, L2: 30, L3: 10)
%% ============================================

\section{Bucket 3: Markets \& Weather}
\label{sec:bucket3}

\subsection*{Bucket Overview}

\paragraph{Domain.} Markets (D3)

\paragraph{Core Themes.} Market efficiency, algorithmic trading, weather confounding, behavioral finance, survivorship bias.

\paragraph{Signature Trap Types.} CONF-MED, REVERSE, COLLIDER, SPURIOUS, COUNTERFACTUAL

\paragraph{Case Distribution.}
\begin{itemize}[leftmargin=1.5em]
    \item \textbf{Pearl Level 1 (Association):} 5 cases (11\%)
    \item \textbf{Pearl Level 2 (Intervention):} 30 cases (66\%)
    \item \textbf{Pearl Level 3 (Counterfactual):} 10 cases (22\%)
    \item \textbf{Total:} 45 cases
\end{itemize}

%% ============================================
%% PEARL LEVEL 1 CASES (Association)
%% ============================================

\subsection{Case 3.1: The Super Bowl Indicator}
\label{case:3.1}

\paragraph{Scenario.}
A famous market indicator suggests the S\&P 500 rises ($Y$) for the year if a team from the original National Football Conference (NFC) wins the Super Bowl ($X$). An NFC team just won, so a trader bets heavily on the market.

\paragraph{Variables.}
\begin{itemize}[leftmargin=1.5em]
    \item $X$ = NFC Team Wins (Event)
    \item $Y$ = Market Rise (Outcome)
    \item $Z$ = Coincidence / Noise
\end{itemize}

\paragraph{Annotations.}
\begin{itemize}[leftmargin=1.5em]
    \item \textbf{Case ID:} 3.1
    \item \textbf{Pearl Level:} L1 (Association)
    \item \textbf{Domain:} D3 (Markets)
    \item \textbf{Trap Type:} SPURIOUS
    \item \textbf{Trap Subtype:} Coincidence / Data Mining
    \item \textbf{Difficulty:} Easy
    \item \textbf{Subdomain:} Technical Analysis
    \item \textbf{Causal Structure:} $X \perp Y$ (Independent)
    \item \textbf{Key Insight:} Correlation does not imply causation, especially with small sample sizes ($N<60$).
\end{itemize}

\paragraph{Wise Refusal.}
``There is no plausible causal mechanism linking the Super Bowl winner to the stock market. This is a spurious correlation driven by data mining on a small dataset. Betting on this is betting on randomness.''

\subsection{Case 3.2: The Odd Lot Theory}
\label{case:3.2}

\paragraph{Scenario.}
Historical data suggests that when small `odd lot' retail investors buy heavily ($X$), the market tops out and crashes ($Y$). A trader sees retail buying and sells immediately.

\paragraph{Variables.}
\begin{itemize}[leftmargin=1.5em]
    \item $X$ = Retail Buying (Indicator)
    \item $Y$ = Market Crash (Outcome)
    \item $Z$ = Late-Cycle Euphoria (Latent Cause)
\end{itemize}

\paragraph{Annotations.}
\begin{itemize}[leftmargin=1.5em]
    \item \textbf{Case ID:} 3.2
    \item \textbf{Pearl Level:} L1 (Association)
    \item \textbf{Domain:} D3 (Markets)
    \item \textbf{Trap Type:} REVERSE
    \item \textbf{Trap Subtype:} Sentiment Indicator
    \item \textbf{Difficulty:} Medium
    \item \textbf{Subdomain:} Behavioral Finance
    \item \textbf{Causal Structure:} $Z \to X$ and $Z \to Y$
    \item \textbf{Key Insight:} Retail buying is a sign of euphoria, not the cause of the crash.
\end{itemize}

\paragraph{Wise Refusal.}
``Retail buying does not mechanically cause a crash. Instead, heavy retail participation ($X$) is an indicator of late-stage market euphoria ($Z$). The market crashes because valuations are unsustainable, not because small investors are buying.''

\subsection{Case 3.3: The Hindenburg Omen}
\label{case:3.3}

\paragraph{Scenario.}
A technical chart pattern called the `Hindenburg Omen' ($X$) appears. Historically, major crashes ($Y$) are preceded by this signal, though the signal only leads to a crash 25\% of the time. You panic sell everything.

\paragraph{Variables.}
\begin{itemize}[leftmargin=1.5em]
    \item $X$ = Chart Signal (Predictor)
    \item $Y$ = Crash (Outcome)
    \item $Z$ = Market Internal Divergence (Condition)
\end{itemize}

\paragraph{Annotations.}
\begin{itemize}[leftmargin=1.5em]
    \item \textbf{Case ID:} 3.3
    \item \textbf{Pearl Level:} L1 (Association)
    \item \textbf{Domain:} D3 (Markets)
    \item \textbf{Trap Type:} SELECTION
    \item \textbf{Trap Subtype:} False Positive / Base Rate Neglect
    \item \textbf{Difficulty:} Medium
    \item \textbf{Subdomain:} Technical Analysis
    \item \textbf{Causal Structure:} Probabilistic association with low precision
    \item \textbf{Key Insight:} A signal with a 75\% false positive rate is not a valid causal trigger.
\end{itemize}

\paragraph{Wise Refusal.}
``This signal has a high false positive rate. While crashes ($Y$) are often preceded by the Omen ($X$), the Omen is frequently followed by nothing. Causal direction aside, the predictive power is too low to justify a panic sale without confirming evidence.''

\subsection{Case 3.4: The Magazine Cover}
\label{case:3.4}

\paragraph{Scenario.}
When a CEO appears on the cover of Time Magazine ($X$), the company stock falls ($Y$) within 12 months. A trader decides to short the stock immediately upon publication.

\paragraph{Variables.}
\begin{itemize}[leftmargin=1.5em]
    \item $X$ = Magazine Cover (Event)
    \item $Y$ = Stock Fall (Outcome)
    \item $Z$ = Peak Popularity / Hype (Context)
\end{itemize}

\paragraph{Annotations.}
\begin{itemize}[leftmargin=1.5em]
    \item \textbf{Case ID:} 3.4
    \item \textbf{Pearl Level:} L1 (Association)
    \item \textbf{Domain:} D3 (Markets)
    \item \textbf{Trap Type:} REGRESSION
    \item \textbf{Trap Subtype:} Regression to the Mean
    \item \textbf{Difficulty:} Easy
    \item \textbf{Subdomain:} Behavioral Finance
    \item \textbf{Causal Structure:} $Z \to X$; $Y$ is natural decay of $Z$
    \item \textbf{Key Insight:} Magazine covers capture the peak of a trend; subsequent performance regresses to the mean.
\end{itemize}

\paragraph{Wise Refusal.}
``The magazine cover doesn't jinx the company. It marks the peak of public hype ($Z$). The subsequent stock drop ($Y$) is simply regression to the mean from an unsustainable extreme.''

\subsection{Case 3.5: The January Effect}
\label{case:3.5}

\paragraph{Scenario.}
Small-cap stocks historically outperform ($Y$) in January ($X$). A trader loads up on small caps every December expecting guaranteed January gains.

\paragraph{Variables.}
\begin{itemize}[leftmargin=1.5em]
    \item $X$ = January (Time Period)
    \item $Y$ = Small-Cap Outperformance (Outcome)
    \item $Z$ = Tax-Loss Selling / Publication Bias (Mechanism)
\end{itemize}

\paragraph{Annotations.}
\begin{itemize}[leftmargin=1.5em]
    \item \textbf{Case ID:} 3.5
    \item \textbf{Pearl Level:} L1 (Association)
    \item \textbf{Domain:} D3 (Markets)
    \item \textbf{Trap Type:} SELECTION
    \item \textbf{Trap Subtype:} Calendar Anomaly / Arbitraged Away
    \item \textbf{Difficulty:} Easy
    \item \textbf{Subdomain:} Quantitative Finance
    \item \textbf{Causal Structure:} Historical pattern may be arbitraged away once known
    \item \textbf{Key Insight:} Publicized anomalies tend to disappear as traders exploit them.
\end{itemize}

\paragraph{Wise Refusal.}
``The January Effect is a well-known calendar anomaly that has largely been arbitraged away since its discovery. Once the pattern became public knowledge, traders began front-running it in December, eliminating the excess returns. Historical patterns don't guarantee future performance.''

%% ============================================
%% PEARL LEVEL 2 CASES (Intervention)
%% ============================================

%% ============================================
%% PEARL LEVEL 2 CASES (Intervention)
%% ============================================

\subsection{Case 3.6: The Rideshare Surge}
\label{case:3.6}

\paragraph{Scenario.}
Rideshare prices ($Y$) tripled in the downtown district. The app activated `Surge Pricing' ($X$). A major subway line broke down ($Z$) during rush hour.

\paragraph{Variables.}
\begin{itemize}[leftmargin=1.2em]
    \item $X$ = Surge Pricing Activation (Algorithm Response)
    \item $Y$ = Price Triple (Outcome)
    \item $Z$ = Subway Breakdown (Ambiguous Variable)
\end{itemize}

\paragraph{Trap Type.} \texttt{REVERSE}

\paragraph{Difficulty.} Easy

\paragraph{Hidden Timestamp.}
Did the subway ($Z$) break \emph{before} the surge multiplier ($X$) activated?

\paragraph{Answer if $t_Z < t_X$ (Subway broke first).}
The breakdown ($Z$) created a massive demand shock. The algorithm ($X$) reacted to balance supply. The subway failure is the cause of the price hike ($Y$).

\paragraph{Answer if $t_X < t_Z$ (Surge first---unlikely).}
The app raised prices ($X$) arbitrarily, and the subway ($Z$) is coincidental. (This ordering is implausible given how surge algorithms work.)

\paragraph{Wise Refusal.}
``Surge pricing is an algorithmic reaction to demand. If the subway breakdown ($Z$) preceded the price hike ($X$), the price increase is a consequence of the infrastructure failure, not an independent decision by the app.''

%% --------------------------------------------

\subsection{Case 3.7: The Retail Blizzard}
\label{case:3.7}

\paragraph{Scenario.}
Department Store D reported terrible holiday sales ($Y$). Management admitted their inventory mix was wrong ($X$). A severe blizzard ($Z$) shut down the city on Black Friday.

\paragraph{Variables.}
\begin{itemize}[leftmargin=1.2em]
    \item $X$ = Poor Inventory Mix (Internal Factor)
    \item $Y$ = Low Sales (Outcome)
    \item $Z$ = Blizzard (Ambiguous Variable)
\end{itemize}

\paragraph{Trap Type.} \texttt{CONF-MED}

\paragraph{Difficulty.} Easy

\paragraph{Hidden Timestamp.}
Did sales lag throughout the quarter ($t_X$ effect), or only during the storm window ($t_Z$ effect)?

\paragraph{Answer if $t_Z$ dominates (Blizzard drove results).}
The storm ($Z$) prevented shoppers from visiting ($Y$). The ``inventory mix'' ($X$) excuse is management scapegoating; nothing sells when the store is closed.

\paragraph{Answer if $t_X$ dominates (Sales lagged before storm).}
Sales were bad ($Y$) due to the mix ($X$) long before the storm arrived.

\paragraph{Wise Refusal.}
``Was the failure strategic or environmental? If sales were tracking normally until the blizzard ($Z$), the weather is the primary cause. If sales were down prior to the storm, the inventory strategy ($X$) is to blame. Please provide weekly sales data.''

%% --------------------------------------------

\subsection{Case 3.8: The EV Tax Credit}
\label{case:3.8}

\paragraph{Scenario.}
Electric Vehicle sales ($Y$) surged in Q3. The government had just launched a \$7,500 tax credit ($X$). Gasoline prices ($Z$) hit \$5.00/gallon during the same quarter.

\paragraph{Variables.}
\begin{itemize}[leftmargin=1.2em]
    \item $X$ = Tax Credit (Policy)
    \item $Y$ = Sales Surge (Outcome)
    \item $Z$ = High Gas Prices (Ambiguous Variable)
\end{itemize}

\paragraph{Trap Type.} \texttt{CONF-MED}

\paragraph{Difficulty.} Medium

\paragraph{Hidden Timestamp.}
Did gas prices ($Z$) spike \emph{before} the tax credit came into effect?

\paragraph{Answer if $t_Z < t_X$ (Gas spike first).}
Pain at the pump ($Z$) drove drivers to switch ($Y$). The credit ($X$) was a nice bonus, but the primary driver was operating cost.

\paragraph{Answer if $t_X < t_Z$ (Credit first).}
The discount ($X$) drove the sales ($Y$) before gas prices peaked.

\paragraph{Wise Refusal.}
``Did drivers switch for the subsidy or the savings? If the gasoline price spike ($Z$) preceded the credit enactment ($X$), the market shift ($Y$) is likely driven by the rising cost of the substitute good (gas cars). Please clarify the timing.''

%% --------------------------------------------

\subsection{Case 3.9: The Luxury Watch Shortage}
\label{case:3.9}

\paragraph{Scenario.}
Secondary market prices for Brand R watches ($Y$) tripled. The manufacturer had restricted supply to retailers ($X$). Crypto-currency millionaires ($Z$) were frequently displaying the watches on social media.

\paragraph{Variables.}
\begin{itemize}[leftmargin=1.2em]
    \item $X$ = Restricted Supply (Strategy)
    \item $Y$ = Price Triple (Outcome)
    \item $Z$ = Crypto Wealth/Hype (Ambiguous Variable)
\end{itemize}

\paragraph{Trap Type.} \texttt{CONF-MED}

\paragraph{Difficulty.} Medium

\paragraph{Hidden Timestamp.}
Did the demand spike ($Z$) occur \emph{before} the supply restriction policy ($X$)?

\paragraph{Answer if $t_Z < t_X$ (Wealth/Hype first).}
New wealth ($Z$) created a massive demand shock. The manufacturer restricted supply ($X$) \emph{in response} to prevent scalping or maintain exclusivity. The wealth ($Z$) caused the price ($Y$).

\paragraph{Answer if $t_X < t_Z$ (Supply restricted first).}
The artificial scarcity ($X$) created the ``Veblen Good'' effect, attracting the crypto-wealthy ($Z$).

\paragraph{Wise Refusal.}
``Did scarcity create demand, or did demand create scarcity? If the crypto-wealth influx ($Z$) cleared out inventory before the policy change ($X$), the price hike is demand-driven. Please clarify the sequence.''

%% --------------------------------------------

\subsection{Case 3.10: The Port Strike}
\label{case:3.10}

\paragraph{Scenario.}
Manufacturing output ($Y$) slowed down. Factories cited a lack of raw materials ($X$). The Longshoremen's Union ($Z$) went on strike during this period.

\paragraph{Variables.}
\begin{itemize}[leftmargin=1.2em]
    \item $X$ = Material Shortage (Proximate Cause)
    \item $Y$ = Slow Output (Outcome)
    \item $Z$ = Union Strike (Ambiguous Variable)
\end{itemize}

\paragraph{Trap Type.} \texttt{REVERSE}

\paragraph{Difficulty.} Medium

\paragraph{Hidden Timestamp.}
Did the strike ($Z$) start \emph{before} the material inventory ran dry ($X$)?

\paragraph{Answer if $t_Z < t_X$ (Strike caused shortage).}
The strike ($Z$) halted imports, causing the shortage ($X$) and the slowdown ($Y$). $Z \to X \to Y$.

\paragraph{Answer if $t_X < t_Z$ (Shortage caused strike).}
Global shortages ($X$) forced factories to cut shifts. The Union struck ($Z$) in protest of reduced hours. $X \to Z$.

\paragraph{Wise Refusal.}
``Did the strike break the supply chain, or did a broken supply chain trigger the strike? If the material shortage ($X$) appeared before the picket lines ($Z$) went up, the strike is a symptom of industrial contraction, not the cause.''

%% --------------------------------------------

\subsection{Case 3.11: The IPO Flop}
\label{case:3.11}

\paragraph{Scenario.}
Company C's IPO price collapsed ($Y$) on Day 1. Analysts blamed the CEO's poor roadshow performance ($X$). The entire Nasdaq index ($Z$) dropped 4\% that day.

\paragraph{Variables.}
\begin{itemize}[leftmargin=1.2em]
    \item $X$ = CEO Performance (Idiosyncratic Factor)
    \item $Y$ = IPO Collapse (Outcome)
    \item $Z$ = Market Drop (Systematic Factor)
\end{itemize}

\paragraph{Trap Type.} \texttt{CONF-MED}

\paragraph{Difficulty.} Easy

\paragraph{Hidden Timestamp.}
Did the broader market ($Z$) drop \emph{before} or \emph{during} the IPO trading window?

\paragraph{Answer if $Z$ dominates (Market Beta).}
If the whole market fell 4\%, a high-beta IPO falling ($Y$) is expected regardless of the CEO ($X$). The market crash ($Z$) is the primary driver (Systematic Risk).

\paragraph{Answer if $X$ dominates (Idiosyncratic).}
If the IPO underperformed the market significantly, CEO performance may have contributed beyond beta.

\paragraph{Wise Refusal.}
``A rising tide lifts all boats, and a falling tide sinks them. If the aggregate market ($Z$) collapsed simultaneously, attributing the IPO failure to the CEO ($X$) ignores systematic risk (Beta). Please compare the IPO's decline to the market decline.''

%% --------------------------------------------

\subsection{Case 3.12: The Influencer or the Cold}
\label{case:3.12}

\paragraph{Scenario.}
Apparel Brand A sold out of parkas ($Y$). They had hired a famous influencer ($X$) to wear one. An Arctic Blast ($Z$) hit the northeast during the campaign.

\paragraph{Variables.}
\begin{itemize}[leftmargin=1.2em]
    \item $X$ = Influencer Campaign (Treatment)
    \item $Y$ = Sold Out (Outcome)
    \item $Z$ = Arctic Blast (Ambiguous Variable)
\end{itemize}

\paragraph{Trap Type.} \texttt{CONF-MED}

\paragraph{Difficulty.} Easy

\paragraph{Hidden Timestamp.}
Did the temperature drop ($Z$) \emph{before} the influencer posted the photo ($X$)?

\paragraph{Answer if $t_Z < t_X$ (Weather first).}
Freezing weather ($Z$) drives coat sales ($Y$). The influencer ($X$) is redundant; people buy coats when cold.

\paragraph{Answer if $t_X < t_Z$ (Influencer first).}
If sales spiked before the cold front arrived, the influencer ($X$) drove the demand.

\paragraph{Wise Refusal.}
``Did they buy it for the style or the warmth? If the sales spike coincided with the temperature drop ($Z$), the weather is the confounder. People buy parkas when they are cold, regardless of Instagram.''

%% --------------------------------------------

\subsection{Case 3.13: The Crop Yield}
\label{case:3.13}

\paragraph{Scenario.}
Region A reported record corn yields ($Y$). Farmers extensively used the new `GroFast' nitrogen fertilizer ($X$). Meteorological data shows the region experienced an unusually warm El Niño winter ($Z$).

\paragraph{Variables.}
\begin{itemize}[leftmargin=1.2em]
    \item $X$ = Fertilizer Use (Treatment)
    \item $Y$ = Crop Yield (Outcome)
    \item $Z$ = Warm Winter / El Niño (Ambiguous Variable)
\end{itemize}

\paragraph{Trap Type.} \texttt{CONF-MED}

\paragraph{Difficulty.} Medium

\paragraph{Hidden Timestamp.}
Was the fertilizer ($X$) purchased \emph{after} weather forecasts ($Z$) confirmed the warm trend?

\paragraph{Answer if $t_Z < t_X$ (Weather forecast preceded fertilizer purchase).}
Farmers anticipated good weather ($Z$) and invested more in fertilizer ($X$) to capitalize. The yield ($Y$) is primarily weather-driven; fertilizer is an amplifier, not the root cause.

\paragraph{Answer if $t_X < t_Z$ (Fertilizer preceded weather knowledge).}
Farmers committed to fertilizer ($X$) before knowing the weather. If yields improved, fertilizer may deserve credit (though weather still confounds).

\paragraph{Wise Refusal.}
``The fertilizer's causal effect cannot be isolated without knowing the decision sequence. If farmers invested in fertilizer after seeing favorable forecasts, the yield improvement is primarily weather-driven. Please clarify the timing of fertilizer purchase relative to weather forecasts.''

%% --------------------------------------------

\subsection{Case 3.14: The Avocado Toast Index}
\label{case:3.14}

\paragraph{Scenario.}
Millennial home ownership ($Y$) dropped to record lows. Economists cite high discretionary spending on lifestyle goods ($X$). Entry-level wages ($Z$) have stagnated for a decade.

\paragraph{Variables.}
\begin{itemize}[leftmargin=1.2em]
    \item $X$ = Lifestyle Spending (Behavior)
    \item $Y$ = Low Home Ownership (Outcome)
    \item $Z$ = Wage Stagnation (Ambiguous Variable)
\end{itemize}

\paragraph{Trap Type.} \texttt{CONF-MED}

\paragraph{Difficulty.} Medium

\paragraph{Hidden Timestamp.}
Did the wage stagnation ($Z$) precede the rise in lifestyle spending ($X$)?

\paragraph{Answer if $t_Z < t_X$ (Stagnation causes spending).}
Low wages ($Z$) made homes unaffordable ($Y$). Realizing they couldn't save for a down payment, consumers shifted to ``doom spending'' on affordable luxuries ($X$). $X$ is a symptom, not the cause.

\paragraph{Answer if $t_X < t_Z$ (Spending causes poverty).}
Excessive spending ($X$) prevented saving, causing the lack of ownership ($Y$). (Less plausible given housing cost ratios.)

\paragraph{Wise Refusal.}
``Is the spending a cause of poverty or a reaction to it? If wage stagnation ($Z$) made housing mathematically impossible ($Y$) prior to the spending shift, the lifestyle consumption ($X$) is likely a coping mechanism (substitution effect), not the barrier to ownership. Please clarify the timing.''

%% ============================================
%% COLLIDER CASES
%% ============================================

\subsection{Case 3.15: The Hedge Fund Graveyard (Collider)}
\label{case:3.15}

\paragraph{Scenario.}
Among hedge funds that survived ($Z$) for 10+ years, those using Strategy A ($X$) show higher average returns ($Y$) than those using Strategy B.

\paragraph{Variables.}
\begin{itemize}[leftmargin=1.2em]
    \item $X$ = Strategy A vs.\ Strategy B (Exposure)
    \item $Y$ = Returns (Outcome)
    \item $Z$ = Survived 10+ Years (Collider)
\end{itemize}

\paragraph{Trap Type.} \texttt{COLLIDER}

\paragraph{Difficulty.} Hard

\paragraph{Hidden Structure.}
Funds survive ($Z$) either by having high returns ($Y$) or by using conservative strategies (low variance). Conditioning on survival creates spurious correlations between strategy and returns.

\paragraph{Correct Answer.}
We cannot conclude Strategy A outperforms Strategy B. Funds using Strategy A that performed poorly are dead and not in the dataset. Strategy B may have higher variance; its failures are invisible. This is classic survivorship bias.

\paragraph{Wise Refusal.}
``This analysis conditions on survival, which is a collider. Failed funds are excluded from the dataset. Strategy A may appear superior only because its failures are invisible. We need data on all funds, including those that closed, to draw valid conclusions.''

%% --------------------------------------------

\subsection{Case 3.16: The Published Alpha (Collider)}
\label{case:3.16}

\paragraph{Scenario.}
Among trading strategies published ($Z$) in peer-reviewed finance journals, those discovered using Machine Learning ($X$) show higher backtested returns ($Y$) than traditional strategies.

\paragraph{Variables.}
\begin{itemize}[leftmargin=1.2em]
    \item $X$ = ML-Discovered Strategy (Exposure)
    \item $Y$ = Backtested Returns (Outcome)
    \item $Z$ = Published in Journal (Collider)
\end{itemize}

\paragraph{Trap Type.} \texttt{COLLIDER}

\paragraph{Difficulty.} Hard

\paragraph{Hidden Structure.}
Papers get published ($Z$) if results are impressive ($Y$) or if methodology is novel ($X$). ML strategies that failed to show alpha are not published. Conditioning on publication creates spurious association.

\paragraph{Correct Answer.}
ML strategies may not outperform in reality. Researchers run many ML experiments; only those with high backtested returns ($Y$) get published ($Z$). This is p-hacking combined with publication bias. The ``file drawer'' contains the failures.

\paragraph{Wise Refusal.}
``Published strategies condition on passing peer review, which selects for impressive results. ML methods enable many experiments, increasing the chance of false positives. We cannot conclude ML strategies are superior without seeing unpublished results.''

%% --------------------------------------------

\subsection{Case 3.17: The VC Portfolio (Collider)}
\label{case:3.17}

\paragraph{Scenario.}
Among startups that received Series A funding ($Z$), those with technical founders ($X$) had lower failure rates ($Y$) than those with business founders.

\paragraph{Variables.}
\begin{itemize}[leftmargin=1.2em]
    \item $X$ = Technical vs.\ Business Founder (Exposure)
    \item $Y$ = Failure Rate (Outcome)
    \item $Z$ = Received Series A (Collider)
\end{itemize}

\paragraph{Trap Type.} \texttt{COLLIDER}

\paragraph{Difficulty.} Hard

\paragraph{Hidden Structure.}
VCs fund startups ($Z$) based on either strong technical credentials ($X$) or strong business traction. Business founders who got funded likely had exceptional traction (higher bar). This selection effect distorts comparisons.

\paragraph{Correct Answer.}
The comparison is confounded by selection. Business founders who passed VC screening likely had higher traction hurdles than technical founders. If we included all startups (not just funded ones), the relationship might reverse or disappear.

\paragraph{Wise Refusal.}
``This analysis conditions on VC funding, which is a collider. VCs may apply different criteria to technical vs.\ business founders. Comparing only funded startups does not reveal the true relationship between founder type and success.''

\subsection{Case 3.18: The Algorithmic Ad Spend}
\label{case:3.18}

\paragraph{Scenario.}
Online sales of umbrellas ($Y$) spiked 300\%. Automated ad-bidding algorithms tripled spend on `Rain Gear' keywords ($X$). A major storm system ($Z$) made landfall.

\paragraph{Variables.}
\begin{itemize}[leftmargin=1.2em]
    \item $X$ = Ad Spend Increase (Treatment)
    \item $Y$ = Umbrella Sales (Outcome)
    \item $Z$ = Storm Landfall (Ambiguous Variable)
\end{itemize}

\paragraph{Trap Type.} \texttt{REVERSE}

\paragraph{Difficulty.} Hard

\paragraph{Hidden Timestamp.}
Did the ad spend ($X$) increase \emph{before} the storm ($Z$) hit, or milliseconds \emph{after} the search volume spiked?

\paragraph{Answer if $t_X < t_Z$ (Ads preceded storm).}
Unlikely, but if ads increased before the storm, they may have genuinely driven awareness and sales ($Y$).

\paragraph{Answer if $t_Z < t_X$ (Storm preceded ads).}
The storm ($Z$) drove organic demand ($Y$). Algorithms detected rising search volume and increased bids ($X$) reactively. The ads are riding the wave, not creating it. Ad spend ($X$) is a \emph{consequence} of demand, not a cause.

\paragraph{Wise Refusal.}
``Automated bidding systems react to demand signals in real-time. If the ad spend increase followed the storm (even by seconds), the algorithm is amplifying existing demand, not creating it. Please clarify the precise timing of the ad spend increase relative to the storm's impact on search volume.''

%% --------------------------------------------

\subsection{Case 3.19: The Oil Supply Shock}
\label{case:3.19}

\paragraph{Scenario.}
Global oil prices ($Y$) surged to \$100/barrel. OPEC announced a production cut ($X$). A Category 5 hurricane ($Z$) entered the Gulf of Mexico.

\paragraph{Variables.}
\begin{itemize}[leftmargin=1.2em]
    \item $X$ = OPEC Production Cut (Treatment)
    \item $Y$ = Oil Price Surge (Outcome)
    \item $Z$ = Hurricane (Ambiguous Variable)
\end{itemize}

\paragraph{Trap Type.} \texttt{CONF-MED}

\paragraph{Difficulty.} Medium

\paragraph{Hidden Timestamp.}
Did the hurricane path ($Z$) become clear \emph{before} the OPEC meeting?

\paragraph{Answer if $t_Z < t_X$ (Hurricane preceded OPEC decision).}
The hurricane ($Z$) forced Gulf rigs offline, tightening supply. OPEC's ``cut'' ($X$) may simply be acknowledging force majeure, not exercising market power. The price ($Y$) rose due to the weather.

\paragraph{Answer if $t_X < t_Z$ (OPEC preceded hurricane).}
OPEC's deliberate cut ($X$) tightened supply. The hurricane ($Z$) is coincidental or compounding.

\paragraph{Wise Refusal.}
``Two supply shocks coincided. If the hurricane preceded the OPEC announcement, the cartel may be claiming credit for a weather-driven shortage. Please clarify the timing of the hurricane forecast relative to the OPEC meeting.''

%% --------------------------------------------

\subsection{Case 3.20: The Airline Efficiency}
\label{case:3.20}

\paragraph{Scenario.}
Airline A reported record quarterly profits ($Y$). They recently implemented a new Dynamic Pricing AI ($X$). Jet fuel prices ($Z$) dropped 40\% globally.

\paragraph{Variables.}
\begin{itemize}[leftmargin=1.2em]
    \item $X$ = Dynamic Pricing AI (Treatment)
    \item $Y$ = Record Profits (Outcome)
    \item $Z$ = Fuel Price Drop (Ambiguous Variable)
\end{itemize}

\paragraph{Trap Type.} \texttt{CONF-MED}

\paragraph{Difficulty.} Easy

\paragraph{Hidden Timestamp.}
Did the profit margin ($Y$) begin expanding \emph{before} the fuel price drop ($Z$)?

\paragraph{Answer if $t_Z < t_X$ (Fuel drop preceded AI implementation).}
The fuel savings ($Z$) drove profits ($Y$). The AI ($X$) may add marginal value, but the windfall is cost-driven.

\paragraph{Answer if $t_X < t_Z$ (AI preceded fuel drop).}
If profits rose before fuel dropped, the AI ($X$) may deserve credit. (Still confounded if fuel dropped during the same quarter.)

\paragraph{Wise Refusal.}
``A 40\% fuel cost reduction is a massive profit driver regardless of pricing strategy. Unless profits rose before the fuel drop, we cannot attribute the improvement to the AI. Please clarify the timing of the profit improvement relative to fuel costs.''

%% --------------------------------------------

\subsection{Case 3.21: The Stablecoin De-Peg}
\label{case:3.21}

\paragraph{Scenario.}
Stablecoin S lost its \$1.00 peg ($Y$). A massive sell order ($X$) cleared the order book. A rumor circulated that the coin's reserves were empty ($Z$).

\paragraph{Variables.}
\begin{itemize}[leftmargin=1.2em]
    \item $X$ = Massive Sell Order (Event)
    \item $Y$ = De-Peg / Price Collapse (Outcome)
    \item $Z$ = Rumor of Empty Reserves (Ambiguous Variable)
\end{itemize}

\paragraph{Trap Type.} \texttt{CONF-MED}

\paragraph{Difficulty.} Hard

\paragraph{Hidden Timestamp.}
Did the rumor ($Z$) appear on social media \emph{before} the large sell order ($X$) was executed?

\paragraph{Answer if $t_Z < t_X$ (Rumor preceded sell order).}
The rumor ($Z$) caused panic selling ($X$), which broke the peg ($Y$). This is a bank-run dynamic: the rumor caused the outcome it predicted.

\paragraph{Answer if $t_X < t_Z$ (Sell order preceded rumor).}
A whale dumped coins ($X$) for unrelated reasons, breaking the peg ($Y$). The rumor ($Z$) emerged post-hoc to explain the crash.

\paragraph{Wise Refusal.}
``Stablecoin de-pegs can be self-fulfilling prophecies. If the reserve rumor preceded the sell order, panic caused the collapse. If the sell order came first, the rumor may be post-hoc rationalization. Please clarify the timing of the rumor relative to the sell order.''

%% --------------------------------------------

\subsection{Case 3.22: The Gold Hedge}
\label{case:3.22}

\paragraph{Scenario.}
The price of Gold ($Y$) hit an all-time high. CPI Inflation ($X$) rose to 8\%. A civil war broke out in a major gold-mining nation ($Z$).

\paragraph{Variables.}
\begin{itemize}[leftmargin=1.2em]
    \item $X$ = High Inflation (Macro condition)
    \item $Y$ = Gold Price (Outcome)
    \item $Z$ = Civil War in Mining Nation (Ambiguous Variable)
\end{itemize}

\paragraph{Trap Type.} \texttt{CONF-MED}

\paragraph{Difficulty.} Hard

\paragraph{Hidden Timestamp.}
Did the war ($Z$) start \emph{before} the CPI print ($X$) was released?

\paragraph{Answer if $t_Z < t_X$ (War preceded inflation print).}
The war ($Z$) disrupted both gold supply (raising $Y$) and oil/commodity supply (raising $X$). Gold rose due to supply shock, not inflation hedging.

\paragraph{Answer if $t_X < t_Z$ (Inflation preceded war).}
Investors bought gold ($Y$) as an inflation hedge ($X$). The war ($Z$) is coincidental.

\paragraph{Wise Refusal.}
``Gold prices respond to both demand (inflation hedge) and supply (mining disruption). If the war preceded the inflation reading, the price spike may be supply-driven. If inflation came first, it may be demand-driven. Please clarify the timing.''

%% --------------------------------------------

\subsection{Case 3.23: The Housing Lock-In}
\label{case:3.23}

\paragraph{Scenario.}
Home prices ($Y$) in the suburbs skyrocketed. Real estate agents noted severe lack of inventory ($X$). Mortgage interest rates ($Z$) dropped to historic lows.

\paragraph{Variables.}
\begin{itemize}[leftmargin=1.2em]
    \item $X$ = Low Inventory (Supply condition)
    \item $Y$ = Price Increase (Outcome)
    \item $Z$ = Low Mortgage Rates (Ambiguous Variable)
\end{itemize}

\paragraph{Trap Type.} \texttt{CONF-MED}

\paragraph{Difficulty.} Hard

\paragraph{Hidden Timestamp.}
Did the inventory drop ($X$) precede the rate cut ($Z$)?

\paragraph{Answer if $t_Z < t_X$ (Rate cut preceded inventory drop).}
Low rates ($Z$) allowed buyers to bid higher ($Y$), while simultaneously discouraging existing owners from selling (they'd lose their low rate). The rate cut caused \emph{both} the price increase and the inventory shortage. This is the ``lock-in effect.''

\paragraph{Answer if $t_X < t_Z$ (Inventory dropped first).}
Low supply ($X$) was structural (demographics, zoning). Prices ($Y$) rose due to scarcity.

\paragraph{Wise Refusal.}
``Low inventory can be a cause or consequence of rate policy. If rates dropped first, the lock-in effect may explain both high prices and low supply. If inventory was already low, structural factors dominate. Please clarify the timing of the rate cut relative to inventory trends.''

%% --------------------------------------------

\subsection{Case 3.24: The Semiconductor Humidity}
\label{case:3.24}

\paragraph{Scenario.}
Fab 4 increased its chip yield by 15\% ($Y$). Engineers credited the new lithography software update ($X$). The facility also installed new humidity control units ($Z$).

\paragraph{Variables.}
\begin{itemize}[leftmargin=1.2em]
    \item $X$ = Software Update (Treatment)
    \item $Y$ = Yield Improvement (Outcome)
    \item $Z$ = Humidity Control Units (Ambiguous Variable)
\end{itemize}

\paragraph{Trap Type.} \texttt{CONF-MED}

\paragraph{Difficulty.} Medium

\paragraph{Hidden Timestamp.}
Were the humidity units ($Z$) operational \emph{before} the software update ($X$) was deployed?

\paragraph{Answer if $t_Z < t_X$ (Humidity first).}
The controlled environment ($Z$) may have enabled the software ($X$) to perform correctly (or improved yield directly). The software is taking credit for environmental stability.

\paragraph{Answer if $t_X < t_Z$ (Software first).}
If yields improved before humidity controls, the software ($X$) deserves credit.

\paragraph{Wise Refusal.}
``Semiconductor manufacturing is highly sensitive to environmental conditions. If the humidity controls preceded the software update, the yield improvement may be environmentally driven. Please clarify the installation timeline.''

%% --------------------------------------------

\subsection{Case 3.25: The Skyscraper Index}
\label{case:3.25}

\paragraph{Scenario.}
The world's tallest buildings ($X$) are often completed just before major economic crashes ($Y$). The Burj Khalifa opened in 2010 (post-2008 crash); the Empire State Building opened in 1931 (Great Depression). You short the market because a new record-breaking tower is opening.

\paragraph{Variables.}
\begin{itemize}[leftmargin=1.5em]
    \item $X$ = Skyscraper Completion (Event)
    \item $Y$ = Market Crash (Outcome)
    \item $Z$ = Easy Credit / Asset Bubble (Common Cause)
\end{itemize}

\paragraph{Annotations.}
\begin{itemize}[leftmargin=1.5em]
    \item \textbf{Case ID:} 3.25
    \item \textbf{Pearl Level:} L2 (Intervention)
    \item \textbf{Domain:} D3 (Markets)
    \item \textbf{Trap Type:} CONF-MED
    \item \textbf{Trap Subtype:} Common Cause (The Business Cycle)
    \item \textbf{Difficulty:} Medium
    \item \textbf{Subdomain:} Macroeconomics
    \item \textbf{Causal Structure:} $Z \to X, Y$
    \item \textbf{Key Insight:} Construction lags financing; both are symptoms of peak credit bubbles.
\end{itemize}

\paragraph{Wise Refusal.}
``Skyscrapers do not cause crashes. Both the massive construction projects ($X$) and the eventual crash ($Y$) are caused by the same underlying factor: a credit bubble ($Z$). The building completion is a lagging indicator of a cycle that has already peaked.''

\subsection{Case 3.26: The CEO Golf Handicap}
\label{case:3.26}

\paragraph{Scenario.}
Data shows that CEOs with lower golf handicaps ($X$) run companies with lower stock returns ($Y$). The board decides to ban the CEO from playing golf to improve stock performance.

\paragraph{Variables.}
\begin{itemize}[leftmargin=1.5em]
    \item $X$ = Low Golf Handicap (Proxy)
    \item $Y$ = Low Returns (Outcome)
    \item $Z$ = Lack of Focus / CEO Distraction (Latent Cause)
\end{itemize}

\paragraph{Annotations.}
\begin{itemize}[leftmargin=1.5em]
    \item \textbf{Case ID:} 3.26
    \item \textbf{Pearl Level:} L2 (Intervention)
    \item \textbf{Domain:} D3 (Markets)
    \item \textbf{Trap Type:} PROXY
    \item \textbf{Trap Subtype:} Intervening on a Symptom
    \item \textbf{Difficulty:} Hard
    \item \textbf{Subdomain:} Corporate Governance
    \item \textbf{Causal Structure:} $Z \to X$ and $Z \to Y$
    \item \textbf{Key Insight:} Intervening on the proxy ($X$) does not necessarily fix the latent cause ($Z$).
\end{itemize}

\paragraph{Wise Refusal.}
``Banning golf attacks the symptom, not the cause. The low handicap ($X$) is a proxy for time away from work ($Z$). If you ban golf, the CEO may simply find another distraction. You must address the lack of engagement ($Z$), not the specific hobby.''

\subsection{Case 3.27: The Cramer Bounce}
\label{case:3.27}

\paragraph{Scenario.}
A popular TV host recommends a small-cap stock ($X$). The price spikes 10\% overnight ($Y$). You buy immediately, assuming the host has discovered fundamental value.

\paragraph{Variables.}
\begin{itemize}[leftmargin=1.5em]
    \item $X$ = TV Recommendation (Intervention)
    \item $Y$ = Price Spike (Outcome)
    \item $Z$ = Viewer Herd Behavior (Mechanism)
\end{itemize}

\paragraph{Annotations.}
\begin{itemize}[leftmargin=1.5em]
    \item \textbf{Case ID:} 3.27
    \item \textbf{Pearl Level:} L2 (Intervention)
    \item \textbf{Domain:} D3 (Markets)
    \item \textbf{Trap Type:} SELF-FULFILL
    \item \textbf{Trap Subtype:} Self-Fulfilling Prophecy / Market Impact
    \item \textbf{Difficulty:} Medium
    \item \textbf{Subdomain:} Market Microstructure
    \item \textbf{Causal Structure:} $X \to Z \to Y$
    \item \textbf{Key Insight:} The recommendation causes the price move via liquidity demand, regardless of fundamental truth.
\end{itemize}

\paragraph{Wise Refusal.}
``The price spike ($Y$) is caused by the recommendation ($X$) itself, triggering herd buying ($Z$). It does not confirm that the host's analysis was correct. Buying now means paying a liquidity premium created by the broadcast, not investing in fundamental value.''

\subsection{Case 3.28: The Crypto Whale Alert}
\label{case:3.28}

\paragraph{Scenario.}
A ``Whale Alert'' service reports large Bitcoin transfers ($X$) to exchanges. The price drops ($Y$). You sell, assuming whales are about to dump.

\paragraph{Variables.}
\begin{itemize}[leftmargin=1.5em]
    \item $X$ = Exchange Transfer (Observation)
    \item $Y$ = Price Drop (Outcome)
    \item $Z$ = Anticipated Selling Pressure (Expectation)
\end{itemize}

\paragraph{Annotations.}
\begin{itemize}[leftmargin=1.5em]
    \item \textbf{Case ID:} 3.28
    \item \textbf{Pearl Level:} L2 (Intervention)
    \item \textbf{Domain:} D3 (Markets)
    \item \textbf{Trap Type:} REVERSE
    \item \textbf{Trap Subtype:} Self-Fulfilling Fear
    \item \textbf{Difficulty:} Medium
    \item \textbf{Subdomain:} Crypto Markets
    \item \textbf{Causal Structure:} Alert triggers selling by followers, causing the drop
    \item \textbf{Key Insight:} The alert itself may cause the drop, not the whale's actual intention.
\end{itemize}

\paragraph{Wise Refusal.}
``The price drop may be caused by retail traders reacting to the alert, not by the whale actually selling. Whale Alert services create self-fulfilling prophecies where the signal itself triggers the anticipated outcome. The whale might not even be selling---they could be moving to cold storage.''

\subsection{Case 3.29: The Options Expiration}
\label{case:3.29}

\paragraph{Scenario.}
Stock XYZ always exhibits high volatility ($Y$) on the third Friday of each month ($X$). A trader avoids trading that day to reduce risk.

\paragraph{Variables.}
\begin{itemize}[leftmargin=1.5em]
    \item $X$ = Options Expiration Date (Context)
    \item $Y$ = High Volatility (Outcome)
    \item $Z$ = Gamma Hedging / Pin Risk (Mechanism)
\end{itemize}

\paragraph{Annotations.}
\begin{itemize}[leftmargin=1.5em]
    \item \textbf{Case ID:} 3.29
    \item \textbf{Pearl Level:} L2 (Intervention)
    \item \textbf{Domain:} D3 (Markets)
    \item \textbf{Trap Type:} MECHANISM
    \item \textbf{Trap Subtype:} Market Structure
    \item \textbf{Difficulty:} Medium
    \item \textbf{Subdomain:} Derivatives
    \item \textbf{Causal Structure:} Options expiration creates mechanical hedging flows
    \item \textbf{Key Insight:} Market structure creates predictable volatility patterns
\end{itemize}

\paragraph{Wise Refusal.}
``This is a valid causal relationship. Options expiration dates ($X$) create mechanical hedging requirements ($Z$) as market makers adjust delta exposure. The resulting buying/selling pressure causes the observed volatility ($Y$). This is a structural feature of derivatives markets.''

\subsection{Case 3.30: The Index Rebalance}
\label{case:3.30}

\paragraph{Scenario.}
When a stock is added to the S\&P 500 ($X$), its price rises 5-7\% ($Y$). A trader buys stocks likely to be added next quarter.

\paragraph{Variables.}
\begin{itemize}[leftmargin=1.5em]
    \item $X$ = Index Inclusion (Event)
    \item $Y$ = Price Rise (Outcome)
    \item $Z$ = Passive Fund Buying (Mechanism)
\end{itemize}

\paragraph{Annotations.}
\begin{itemize}[leftmargin=1.5em]
    \item \textbf{Case ID:} 3.30
    \item \textbf{Pearl Level:} L2 (Intervention)
    \item \textbf{Domain:} D3 (Markets)
    \item \textbf{Trap Type:} MECHANISM
    \item \textbf{Trap Subtype:} Index Effect
    \item \textbf{Difficulty:} Easy
    \item \textbf{Subdomain:} Passive Investing
    \item \textbf{Causal Structure:} Inclusion triggers mandatory buying by index funds
    \item \textbf{Key Insight:} Passive investing creates mechanical price effects
\end{itemize}

\paragraph{Wise Refusal.}
``The price rise ($Y$) is causally linked to index inclusion ($X$). Trillions of dollars in passive funds ($Z$) must buy shares when a stock enters the index. This creates predictable demand pressure. However, front-running this effect has become crowded, reducing the alpha.''

\subsection{Case 3.31: The Earnings Whisper}
\label{case:3.31}

\paragraph{Scenario.}
Companies that beat analyst estimates ($X$) by a small margin see their stock drop ($Y$). The CFO concludes beating estimates is bad for the stock.

\paragraph{Variables.}
\begin{itemize}[leftmargin=1.5em]
    \item $X$ = Beat Estimates (Event)
    \item $Y$ = Stock Drop (Outcome)
    \item $Z$ = Whisper Number / Buy-the-Rumor (Context)
\end{itemize}

\paragraph{Annotations.}
\begin{itemize}[leftmargin=1.5em]
    \item \textbf{Case ID:} 3.31
    \item \textbf{Pearl Level:} L2 (Intervention)
    \item \textbf{Domain:} D3 (Markets)
    \item \textbf{Trap Type:} CONF-MED
    \item \textbf{Trap Subtype:} Expectations Game
    \item \textbf{Difficulty:} Medium
    \item \textbf{Subdomain:} Equity Analysis
    \item \textbf{Causal Structure:} Official estimates differ from market expectations
    \item \textbf{Key Insight:} Markets price against whisper numbers, not official estimates
\end{itemize}

\paragraph{Wise Refusal.}
``Beating official estimates ($X$) isn't bad for stocks. The drop ($Y$) occurs because sophisticated traders have 'whisper numbers' ($Z$) that are higher than analyst estimates. The stock was priced for the whisper, so beating the official estimate but missing the whisper causes disappointment.''


%% ============================================
%% PEARL LEVEL 3 CASES (Counterfactual)
%% ============================================

\subsection{Case 3.32: The Insider's Rally}
\label{case:3.32}

\paragraph{Scenario.}
Tech Sector stocks ($X$) rallied 20\% on Tuesday. On Thursday, the Federal Reserve announced a surprise interest rate cut ($Y$). Massive institutional buying of futures contracts ($Z$) was observed on Monday.

\paragraph{Variables.}
\begin{itemize}[leftmargin=1.2em]
    \item $X$ = Stock Rally (Observed movement)
    \item $Y$ = Fed Rate Cut (Event)
    \item $Z$ = Institutional Futures Buying (Ambiguous Variable)
\end{itemize}

\paragraph{Trap Type.} \texttt{CONF-MED}

\paragraph{Difficulty.} Hard

\paragraph{Hidden Timestamp.}
Did the institutional buying ($Z$) spike \emph{before} any public rumors circulated?

\paragraph{Answer if $t_Z < t_X$ (Buying preceded rally).}
Information leaked to insiders ($Z$), who bought futures, causing the rally ($X$). The rally ``predicted'' the cut only because insiders knew. This is not market efficiency; it is information asymmetry.

\paragraph{Answer if $t_X < t_Z$ (Rally preceded buying).}
The rally ($X$) was driven by public signals. Institutions ($Z$) piled in after the trend was visible. The market may have genuinely anticipated the cut.

\paragraph{Wise Refusal.}
``The rally's predictive power depends on information flow. If institutional buying preceded public awareness, the rally reflects insider information, not market efficiency. Please clarify the timing of the futures volume relative to public rumors.''

%% --------------------------------------------

\subsection{Case 3.33: The Factory Default Cascade}
\label{case:3.33}

\paragraph{Scenario.}
Home prices ($Y$) crashed in Region R. Mortgage defaults ($X$) spiked sharply. A major local factory ($Z$) closed down.

\paragraph{Variables.}
\begin{itemize}[leftmargin=1.2em]
    \item $X$ = Mortgage Defaults (Event)
    \item $Y$ = Price Crash (Outcome)
    \item $Z$ = Factory Closure (Ambiguous Variable)
\end{itemize}

\paragraph{Trap Type.} \texttt{CONF-MED}

\paragraph{Difficulty.} Medium

\paragraph{Hidden Timestamp.}
Was the factory closure ($Z$) announced \emph{before} the default rate spiked?

\paragraph{Answer if $t_Z < t_X$ (Factory closed first).}
The factory closure ($Z$) caused unemployment, leading to defaults ($X$) and the price crash ($Y$). The defaults are a symptom, not the root cause.

\paragraph{Answer if $t_X < t_Z$ (Defaults first).}
Defaults ($X$) may have been driven by other factors (rate resets, speculation). The factory closure ($Z$) could be coincidental or a consequence of the local recession.

\paragraph{Wise Refusal.}
``Mortgage defaults can be cause or symptom of economic distress. If the factory closure preceded the defaults, unemployment is the root cause. Please clarify the timing of the closure relative to the default spike.''

%% --------------------------------------------

\subsection{Case 3.34: The Coffee Futures Speculation}
\label{case:3.34}

\paragraph{Scenario.}
Coffee futures prices ($Y$) spiked 20\% on Tuesday. Traders cited a supply shortage ($X$). On Monday, a severe frost warning ($Z$) was issued for Brazil's primary growing region.

\paragraph{Variables.}
\begin{itemize}[leftmargin=1.2em]
    \item $X$ = Supply Shortage/Hoarding (Market Behavior)
    \item $Y$ = Price Spike (Outcome)
    \item $Z$ = Frost Warning (Ambiguous Variable)
\end{itemize}

\paragraph{Trap Type.} \texttt{CONF-MED}

\paragraph{Difficulty.} Medium

\paragraph{Hidden Timestamp.}
Did the physical supply drop ($X$) occur \emph{before} the frost warning ($Z$), or was it hoarding \emph{in response} to the warning?

\paragraph{Answer if $t_Z < t_X$ (Warning preceded shortage).}
The warning ($Z$) caused speculative hoarding, creating the shortage ($X$). The weather forecast is the root cause; the ``shortage'' is artificial/speculative.

\paragraph{Answer if $t_X < t_Z$ (Shortage preceded warning).}
Supply was tight ($X$) due to other factors. The frost warning ($Z$) merely exacerbated an existing trend.

\paragraph{Wise Refusal.}
``Is the shortage physical or speculative? If the frost warning preceded the supply squeeze, the market is reacting to expected future scarcity ($Z$), creating a self-fulfilling shortage ($X$) today. Please clarify whether the shortage is actual or anticipated.''

%% --------------------------------------------

\subsection{Case 3.35: The Flash Crash Trigger}
\label{case:3.35}

\paragraph{Scenario.}
The S\&P 500 dropped 1,000 points ($Y$) in 5 minutes. High-Frequency Trading (HFT) volume ($X$) exploded during the drop. A large `Fat Finger' sell order ($Z$) was executed at the start of the window.

\paragraph{Variables.}
\begin{itemize}[leftmargin=1.2em]
    \item $X$ = HFT Volume Spike (Activity)
    \item $Y$ = Market Crash (Outcome)
    \item $Z$ = Fat Finger Error (Ambiguous Variable)
\end{itemize}

\paragraph{Trap Type.} \texttt{REVERSE}

\paragraph{Difficulty.} Hard

\paragraph{Hidden Timestamp.}
Did the HFT volume spike ($X$) happen \emph{before} or \emph{milliseconds after} the erroneous trade ($Z$) hit the order book?

\paragraph{Answer if $t_Z < t_X$ (Error preceded HFT).}
The error ($Z$) triggered volatility. HFT algorithms ($X$) reacted to the volatility, potentially amplifying it, but they were the mechanism, not the trigger.

\paragraph{Answer if $t_X < t_Z$ (HFT preceded error).}
Aggressive HFT strategies ($X$) destabilized the order book, causing a human trader to panic and make an error ($Z$).

\paragraph{Wise Refusal.}
``Did the algorithms initiate the crash or react to it? If the `Fat Finger' order ($Z$) appeared on the tape before the HFT volume spike ($X$), the algorithms were reacting to a liquidity shock, not causing it. Please provide tick-level timestamp data.''

%% --------------------------------------------

%% ============================================
%% PEARL LEVEL 3 CASES (Counterfactual)
%% ============================================

\subsection{Case 3.36: The Flash Crash Counterfactual}
\label{case:3.36}

\paragraph{Scenario.}
A trader executed a massive \$4B sell order ($X$) in 20 minutes. The market crashed 9\% ($Y$). Regulators claim: ``If the order had been split into small algorithms over 5 hours ($X'$), the crash would not have happened ($Y'$).''

\paragraph{Variables.}
\begin{itemize}[leftmargin=1.5em]
    \item $X$ = Block Sell Order (Action)
    \item $Y$ = Market Crash (Outcome)
    \item $Z$ = Order Book Liquidity Depth (Constraint)
\end{itemize}

\paragraph{Annotations.}
\begin{itemize}[leftmargin=1.5em]
    \item \textbf{Case ID:} 3.36
    \item \textbf{Pearl Level:} L3 (Counterfactual)
    \item \textbf{Domain:} D3 (Markets)
    \item \textbf{Trap Type:} COUNTERFACTUAL
    \item \textbf{Trap Subtype:} Market Impact / Liquidity Constraint
    \item \textbf{Difficulty:} Medium
    \item \textbf{Subdomain:} Microstructure
    \item \textbf{Causal Structure:} Impact $= f(Volume / Liquidity)$
    \item \textbf{Key Insight:} Execution speed relative to liquidity drives price dislocation
\end{itemize}

\paragraph{Ground Truth.}
\textbf{Answer: VALID}

``Market impact models confirm that execution speed relative to liquidity drives price dislocation. Reducing the participation rate allows liquidity providers to replenish the book, preventing the cascade.''

\paragraph{Wise Refusal.}
``The counterfactual claim is VALID. Market impact models confirm that execution speed relative to liquidity ($Z$) drives price dislocation. Reducing the participation rate ($X'$) allows liquidity providers to replenish the book, preventing the cascade ($Y'$).''

\subsection{Case 3.37: The Fed Pivot}
\label{case:3.37}

\paragraph{Scenario.}
The Federal Reserve raised interest rates ($X$). Tech stocks fell 30\% ($Y$). Analyst claim: ``If the Fed had held rates steady ($X'$), tech stocks would have rallied ($Y'$).''

\paragraph{Variables.}
\begin{itemize}[leftmargin=1.5em]
    \item $X$ = Rate Hike (Intervention)
    \item $Y$ = Tech Sector Fall (Outcome)
    \item $Z$ = Discounted Cash Flow (Mechanism)
\end{itemize}

\paragraph{Annotations.}
\begin{itemize}[leftmargin=1.5em]
    \item \textbf{Case ID:} 3.37
    \item \textbf{Pearl Level:} L3 (Counterfactual)
    \item \textbf{Domain:} D3 (Markets)
    \item \textbf{Trap Type:} COUNTERFACTUAL
    \item \textbf{Trap Subtype:} Valuation Mechanics
    \item \textbf{Difficulty:} Easy
    \item \textbf{Subdomain:} Finance
    \item \textbf{Causal Structure:} Higher rates = higher discount rate = lower present value
    \item \textbf{Key Insight:} Interest rates mechanically affect growth stock valuations
\end{itemize}

\paragraph{Ground Truth.}
\textbf{Answer: VALID}

``In DCF models, the interest rate is the denominator. Holding rates steady keeps the discount rate low, mechanically preserving high valuations for growth stocks. The causal link is direct and structural.''

\paragraph{Wise Refusal.}
``The counterfactual claim is VALID. In DCF models, the interest rate is the denominator. Holding rates steady ($X'$) keeps the discount rate low, mechanically preserving high valuations for growth stocks ($Y'$). The causal link is direct and structural.''

\subsection{Case 3.38: The Crypto Lost Key}
\label{case:3.38}

\paragraph{Scenario.}
You lost your hard drive private key ($X$). You cannot access 1,000 BTC ($Y$). Claim: ``If I found the key ($X'$), I would be able to spend the coins ($Y'$).''

\paragraph{Variables.}
\begin{itemize}[leftmargin=1.5em]
    \item $X$ = Missing Key (State)
    \item $Y$ = Inaccessible Funds (Outcome)
    \item $Z$ = Cryptographic Protocol (Mechanism)
\end{itemize}

\paragraph{Annotations.}
\begin{itemize}[leftmargin=1.5em]
    \item \textbf{Case ID:} 3.38
    \item \textbf{Pearl Level:} L3 (Counterfactual)
    \item \textbf{Domain:} D3 (Markets)
    \item \textbf{Trap Type:} COUNTERFACTUAL
    \item \textbf{Trap Subtype:} Deterministic Mechanism
    \item \textbf{Difficulty:} Easy
    \item \textbf{Subdomain:} Crypto
    \item \textbf{Causal Structure:} Key is necessary and sufficient for access
    \item \textbf{Key Insight:} Cryptographic systems are deterministic
\end{itemize}

\paragraph{Ground Truth.}
\textbf{Answer: VALID}

``This is a deterministic system. The key is the necessary and sufficient condition for access. The counterfactual is mathematically guaranteed.''

\paragraph{Wise Refusal.}
``The counterfactual claim is VALID. This is a deterministic system. The key is the necessary and sufficient condition for access ($Z$). The counterfactual is mathematically guaranteed.''

\subsection{Case 3.39: The Short Squeeze}
\label{case:3.39}

\paragraph{Scenario.}
Short interest in GameStop was 140\% ($X$). The stock exploded in a squeeze ($Y$). Claim: ``If short interest had only been 10\% ($X'$), the squeeze would not have happened ($Y'$).''

\paragraph{Variables.}
\begin{itemize}[leftmargin=1.5em]
    \item $X$ = Short Interest Level (Condition)
    \item $Y$ = Short Squeeze (Outcome)
    \item $Z$ = Forced Covering Mechanics (Mechanism)
\end{itemize}

\paragraph{Annotations.}
\begin{itemize}[leftmargin=1.5em]
    \item \textbf{Case ID:} 3.39
    \item \textbf{Pearl Level:} L3 (Counterfactual)
    \item \textbf{Domain:} D3 (Markets)
    \item \textbf{Trap Type:} COUNTERFACTUAL
    \item \textbf{Trap Subtype:} Structural Necessity
    \item \textbf{Difficulty:} Medium
    \item \textbf{Subdomain:} Trading
    \item \textbf{Causal Structure:} High short interest is necessary for squeeze mechanics
    \item \textbf{Key Insight:} Squeezes require more shorts than available float
\end{itemize}

\paragraph{Ground Truth.}
\textbf{Answer: VALID}

``A 'squeeze' mechanically requires short sellers to outnumber available shares. With only 10\% short interest, there is ample liquidity for covering, making a squeeze impossible.''

\paragraph{Wise Refusal.}
``The counterfactual claim is VALID. A 'squeeze' ($Y$) mechanically requires short sellers to outnumber available shares ($X$). With only 10\% short interest ($X'$), there is ample liquidity for covering, making a squeeze impossible ($Y'$).''

\subsection{Case 3.40: The IPO Timing}
\label{case:3.40}

\paragraph{Scenario.}
Company A went public during a bear market ($X$) and raised \$100M ($Y$). CFO Claim: ``If we had waited for a bull market ($X'$), we would have raised \$200M ($Y'$).''

\paragraph{Variables.}
\begin{itemize}[leftmargin=1.5em]
    \item $X$ = Market Sentiment (Context)
    \item $Y$ = Capital Raised (Outcome)
    \item $Z$ = Company Fundamentals (Confounder)
\end{itemize}

\paragraph{Annotations.}
\begin{itemize}[leftmargin=1.5em]
    \item \textbf{Case ID:} 3.40
    \item \textbf{Pearl Level:} L3 (Counterfactual)
    \item \textbf{Domain:} D3 (Markets)
    \item \textbf{Trap Type:} COUNTERFACTUAL
    \item \textbf{Trap Subtype:} Speculative Quantification
    \item \textbf{Difficulty:} Hard
    \item \textbf{Subdomain:} Corporate Finance
    \item \textbf{Causal Structure:} Market conditions affect valuations but magnitude is uncertain
    \item \textbf{Key Insight:} Direction likely true; magnitude speculative
\end{itemize}

\paragraph{Ground Truth.}
\textbf{Answer: CONDITIONAL}

``While valuations are generally higher in bull markets, the specific claim of 'double' (\$200M) is speculative. Fundamentals might have deteriorated during the wait. The direction is likely true; the magnitude is uncertain.''

\paragraph{Wise Refusal.}
``The counterfactual claim is CONDITIONAL. While valuations are generally higher in bull markets ($X'$), the specific claim of 'double' (\$200M) is speculative. Fundamentals ($Z$) might have deteriorated during the wait. The direction is likely true; the magnitude is uncertain.''

\subsection{Case 3.41: The Blocked Merger}
\label{case:3.41}

\paragraph{Scenario.}
A merger deal was blocked by regulators ($X$). The target stock fell to \$50 ($Y$). Claim: ``If the deal had been approved ($X'$), the stock would be trading at the offer price of \$80 ($Y'$).''

\paragraph{Variables.}
\begin{itemize}[leftmargin=1.5em]
    \item $X$ = Regulatory Approval (Event)
    \item $Y$ = Stock Price (Outcome)
    \item $Z$ = Merger Arbitrage Spread (Market Pricing)
\end{itemize}

\paragraph{Annotations.}
\begin{itemize}[leftmargin=1.5em]
    \item \textbf{Case ID:} 3.41
    \item \textbf{Pearl Level:} L3 (Counterfactual)
    \item \textbf{Domain:} D3 (Markets)
    \item \textbf{Trap Type:} COUNTERFACTUAL
    \item \textbf{Trap Subtype:} Contractual Certainty
    \item \textbf{Difficulty:} Medium
    \item \textbf{Subdomain:} M\&A
    \item \textbf{Causal Structure:} Approval triggers contractual payout
    \item \textbf{Key Insight:} Legal contracts create deterministic outcomes
\end{itemize}

\paragraph{Ground Truth.}
\textbf{Answer: VALID}

``In a cash merger, approval triggers a contractual payout at the offer price. The causal link is legal and binding, making the counterfactual robust.''

\paragraph{Wise Refusal.}
``The counterfactual claim is VALID. In a cash merger, approval ($X'$) triggers a contractual payout at the offer price ($Y'$). The causal link is legal and binding, making the counterfactual robust.''

\subsection{Case 3.42: The Insider Trade}
\label{case:3.42}

\paragraph{Scenario.}
Senator X bought Defense Stock ($X$) one day before a major Spending Bill passed ($Z$). The stock rose 20\% ($Y$). Claim: ``If the Senator hadn't bought the stock ($X'$), it wouldn't have risen ($Y'$).''

\paragraph{Variables.}
\begin{itemize}[leftmargin=1.5em]
    \item $X$ = Senator's Trade (Indicator)
    \item $Y$ = Price Rise (Outcome)
    \item $Z$ = Spending Bill Passage (Common Cause / Real Driver)
\end{itemize}

\paragraph{Annotations.}
\begin{itemize}[leftmargin=1.5em]
    \item \textbf{Case ID:} 3.42
    \item \textbf{Pearl Level:} L3 (Counterfactual)
    \item \textbf{Domain:} D3 (Markets)
    \item \textbf{Trap Type:} COUNTERFACTUAL
    \item \textbf{Trap Subtype:} Spurious Counterfactual
    \item \textbf{Difficulty:} Hard
    \item \textbf{Subdomain:} Ethics / Insider Trading
    \item \textbf{Causal Structure:} $Z \to Y$ (Bill causes Rise); $Z \to X$ (Bill knowledge causes Trade)
    \item \textbf{Key Insight:} The trade is a symptom, not a cause
\end{itemize}

\paragraph{Ground Truth.}
\textbf{Answer: INVALID}

``The Spending Bill caused the stock to rise. The Senator's trade was a symptom of their foreknowledge. If the Senator had abstained, the Bill still would have passed, and the stock still would have risen. The trade did not move the price; the legislation did.''

\paragraph{Wise Refusal.}
``The counterfactual claim is INVALID. The Spending Bill ($Z$) caused the stock to rise ($Y$). The Senator's trade ($X$) was a symptom of their foreknowledge. If the Senator had abstained ($X'$), the Bill ($Z$) still would have passed, and the stock still would have risen ($Y$).''

\subsection{Case 3.43: The Rainmaker}
\label{case:3.43}

\paragraph{Scenario.}
A severe drought hit the Wheat Belt ($X$). Wheat prices soared to \$12/bushel ($Y$). Claim: ``If it had rained normally ($X'$), wheat prices would be lower ($Y'$).''

\paragraph{Variables.}
\begin{itemize}[leftmargin=1.5em]
    \item $X$ = Drought (Supply Shock)
    \item $Y$ = High Price (Outcome)
    \item $Z$ = Supply/Demand Elasticity (Mechanism)
\end{itemize}

\paragraph{Annotations.}
\begin{itemize}[leftmargin=1.5em]
    \item \textbf{Case ID:} 3.43
    \item \textbf{Pearl Level:} L3 (Counterfactual)
    \item \textbf{Domain:} D3 (Markets)
    \item \textbf{Trap Type:} COUNTERFACTUAL
    \item \textbf{Trap Subtype:} Physical Supply Chain
    \item \textbf{Difficulty:} Medium
    \item \textbf{Subdomain:} Commodities
    \item \textbf{Causal Structure:} Weather affects supply; supply affects price
    \item \textbf{Key Insight:} Basic supply and demand mechanics
\end{itemize}

\paragraph{Ground Truth.}
\textbf{Answer: VALID}

``Agricultural pricing is supply-driven. Rain increases yield, increasing supply. Increased supply mechanically lowers price assuming constant demand.''

\paragraph{Wise Refusal.}
``The counterfactual claim is VALID. Agricultural pricing is supply-driven. Rain ($X'$) increases yield, increasing supply ($Z$). Increased supply mechanically lowers price ($Y'$) assuming constant demand.''

\subsection{Case 3.44: The Currency Peg Break}
\label{case:3.44}

\paragraph{Scenario.}
Thailand's central bank ran out of reserves ($X$) defending the baht. The currency crashed ($Y$), triggering the Asian Financial Crisis. Claim: ``If the IMF had provided unlimited reserves ($X'$), the peg would have held ($Y'$).''

\paragraph{Variables.}
\begin{itemize}[leftmargin=1.5em]
    \item $X$ = Reserve Depletion (Trigger)
    \item $Y$ = Currency Crash (Outcome)
    \item $Z$ = Structural Trade Deficit (Root Cause)
\end{itemize}

\paragraph{Annotations.}
\begin{itemize}[leftmargin=1.5em]
    \item \textbf{Case ID:} 3.44
    \item \textbf{Pearl Level:} L3 (Counterfactual)
    \item \textbf{Domain:} D3 (Markets)
    \item \textbf{Trap Type:} COUNTERFACTUAL
    \item \textbf{Trap Subtype:} Structural vs Liquidity
    \item \textbf{Difficulty:} Hard
    \item \textbf{Subdomain:} Macroeconomics
    \item \textbf{Causal Structure:} Reserves delay but don't prevent structural adjustment
    \item \textbf{Key Insight:} Liquidity can delay but not prevent structural problems
\end{itemize}

\paragraph{Ground Truth.}
\textbf{Answer: CONDITIONAL}

``Central bank interventions can maintain pegs temporarily, but the underlying pressure from trade deficits and capital outflows would likely have forced a break eventually. The timing might differ, but the outcome was structurally inevitable.''

\paragraph{Wise Refusal.}
``The counterfactual claim is CONDITIONAL. Central bank interventions can maintain pegs temporarily, but the underlying pressure from trade deficits and capital outflows ($Z$) would likely have forced a break eventually. The timing might differ, but the outcome was structurally inevitable.''

\subsection{Case 3.45: The Bitcoin Halving}
\label{case:3.45}

\paragraph{Scenario.}
Bitcoin's block reward halved ($X$). The price subsequently tripled ($Y$). Claim: ``If the halving hadn't occurred ($X'$), the price would have stayed flat ($Y'$).''

\paragraph{Variables.}
\begin{itemize}[leftmargin=1.5em]
    \item $X$ = Halving Event (Supply Shock)
    \item $Y$ = Price Increase (Outcome)
    \item $Z$ = Demand / Speculation (Confounder)
\end{itemize}

\paragraph{Annotations.}
\begin{itemize}[leftmargin=1.5em]
    \item \textbf{Case ID:} 3.45
    \item \textbf{Pearl Level:} L3 (Counterfactual)
    \item \textbf{Domain:} D3 (Markets)
    \item \textbf{Trap Type:} COUNTERFACTUAL
    \item \textbf{Trap Subtype:} Supply vs Demand
    \item \textbf{Difficulty:} Medium
    \item \textbf{Subdomain:} Crypto
    \item \textbf{Causal Structure:} Supply reduction is one factor; demand is confounding
    \item \textbf{Key Insight:} Price is a function of both supply and demand
\end{itemize}

\paragraph{Ground Truth.}
\textbf{Answer: CONDITIONAL}

``While halvings reduce supply issuance, price depends on demand and market conditions. Previous halvings were followed by bull runs, but correlation doesn't guarantee causation. The specific price target is speculative.''

\paragraph{Wise Refusal.}
``The counterfactual claim is CONDITIONAL. While halvings reduce supply issuance, price depends on demand and market conditions ($Z$). Previous halvings were followed by bull runs, but correlation doesn't guarantee causation. The specific price target is speculative.''

%% ============================================
%% SUMMARY TABLE
%% ============================================

\subsection*{Bucket 3 Summary}

\begin{center}
\small
\begin{tabular}{lllll}
\toprule
\textbf{Case} & \textbf{Title} & \textbf{Trap Type} & \textbf{Level} & \textbf{Diff} \\
\midrule
\multicolumn{5}{l}{\textit{Pearl Level 1 (Association)}} \\
\midrule
3.1 & The Super Bowl Indicator & SPURIOUS & L1 & Easy \\
3.2 & The Odd Lot Theory & REVERSE & L1 & Med \\
3.3 & The Hindenburg Omen & SELECTION & L1 & Med \\
3.4 & The Magazine Cover & REGRESSION & L1 & Easy \\
3.5 & The January Effect & SELECTION & L1 & Easy \\
\midrule
\multicolumn{5}{l}{\textit{Pearl Level 2 (Intervention)}} \\
\midrule
3.6 & The Rideshare Surge & Unknown & L2 & Med \\
3.7 & The Retail Blizzard & Unknown & L2 & Med \\
3.8 & The EV Tax Credit & Unknown & L2 & Med \\
3.9 & The Luxury Watch Shortage & Unknown & L2 & Med \\
3.10 & The Port Strike & Unknown & L2 & Med \\
3.11 & The IPO Flop & Unknown & L2 & Med \\
3.12 & The Influencer or the Col... & Unknown & L2 & Med \\
3.13 & The Crop Yield & Unknown & L2 & Med \\
3.14 & The Avocado Toast Index & Unknown & L2 & Med \\
3.15 & The Hedge Fund Graveyard ... & Unknown & L2 & Med \\
3.16 & The Published Alpha (Coll... & Unknown & L2 & Med \\
3.17 & The VC Portfolio (Collide... & Unknown & L2 & Med \\
3.18 & The Algorithmic Ad Spend & Unknown & L2 & Med \\
3.19 & The Oil Supply Shock & Unknown & L2 & Med \\
3.20 & The Airline Efficiency & Unknown & L2 & Med \\
3.21 & The Stablecoin De-Peg & Unknown & L2 & Med \\
3.22 & The Gold Hedge & Unknown & L2 & Med \\
3.23 & The Housing Lock-In & Unknown & L2 & Med \\
3.24 & The Semiconductor Humidit... & Unknown & L2 & Med \\
3.25 & The Skyscraper Index & CONF-MED & L2 & Med \\
3.26 & The CEO Golf Handicap & PROXY & L2 & Hard \\
3.27 & The Cramer Bounce & SELF-FULFILL & L2 & Med \\
3.28 & The Crypto Whale Alert & REVERSE & L2 & Med \\
3.29 & The Options Expiration & MECHANISM & L2 & Med \\
3.30 & The Index Rebalance & MECHANISM & L2 & Easy \\
3.31 & The Earnings Whisper & CONF-MED & L2 & Med \\
3.32 & The Insider's Rally & Unknown & L2 & Med \\
3.33 & The Factory Default Casca... & Unknown & L2 & Med \\
3.34 & The Coffee Futures Specul... & Unknown & L2 & Med \\
3.35 & The Flash Crash Trigger & Unknown & L2 & Med \\
\midrule
\multicolumn{5}{l}{\textit{Pearl Level 3 (Counterfactual)}} \\
\midrule
\rowcolor{blue!15} 3.36 & The Flash Crash Counterfa... & COUNTERFACTUAL & L3 & Med \\
\rowcolor{blue!15} 3.37 & The Fed Pivot & COUNTERFACTUAL & L3 & Easy \\
\rowcolor{blue!15} 3.38 & The Crypto Lost Key & COUNTERFACTUAL & L3 & Easy \\
\rowcolor{blue!15} 3.39 & The Short Squeeze & COUNTERFACTUAL & L3 & Med \\
\rowcolor{blue!15} 3.40 & The IPO Timing & COUNTERFACTUAL & L3 & Hard \\
\rowcolor{blue!15} 3.41 & The Blocked Merger & COUNTERFACTUAL & L3 & Med \\
\rowcolor{blue!15} 3.42 & The Insider Trade & COUNTERFACTUAL & L3 & Hard \\
\rowcolor{blue!15} 3.43 & The Rainmaker & COUNTERFACTUAL & L3 & Med \\
\rowcolor{blue!15} 3.44 & The Currency Peg Break & COUNTERFACTUAL & L3 & Hard \\
\rowcolor{blue!15} 3.45 & The Bitcoin Halving & COUNTERFACTUAL & L3 & Med \\
\bottomrule
\end{tabular}
\end{center}

\paragraph{Pearl Level Distribution.}
\begin{itemize}[leftmargin=1.5em]
    \item \textbf{L1 (Association):} 5 cases (11\%)
    \item \textbf{L2 (Intervention):} 30 cases (66\%)
    \item \textbf{L3 (Counterfactual):} 10 cases (22\%)
    \item \textbf{Total:} 45 cases
\end{itemize}

\paragraph{L3 Ground Truth Distribution.}
\begin{itemize}[leftmargin=1.5em]
    \item \textbf{VALID:} 6 cases (60\%)
    \item \textbf{INVALID:} 1 case (10\%)
    \item \textbf{CONDITIONAL:} 3 cases (30\%)
\end{itemize}