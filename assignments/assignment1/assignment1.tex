\documentclass[11pt]{article}
\usepackage[utf8]{inputenc}
\usepackage{geometry}
\usepackage{amsmath}
\usepackage{amssymb}
\usepackage{enumitem}
\usepackage{hyperref}
\usepackage{xcolor}
\usepackage{booktabs}
\usepackage{longtable}

\geometry{margin=1in}

\title{CS372 Assignment 1: T³ Benchmark Analysis}
\author{CS372: Artificial General Intelligence for Reasoning, Planning, and Decision Making}
\date{Winter 2026}

\begin{document}

\maketitle

\section{Assignment Overview}

This assignment focuses on analyzing and working with the T³ (T-cubed) Benchmark datasets. The T³ Benchmark is designed to test reasoning capabilities across different domains and Pearl's Causality Hierarchy levels (Association, Intervention, and Counterfactual).

\section{Group Structure}

There will be \textbf{20 groups} for this assignment, with approximately 6 students per group (120 students total). Each of the 10 BenchmarkT3-BucketLarge files will be assigned to \textbf{2 groups} for cross-validation purposes.

\subsection{Group Assignments}

The following table shows the group assignments, benchmark files, their corresponding categories, and dataset information:

\begin{table}[h]
\centering
\begin{tabular}{cllcc}
\toprule
\textbf{Group} & \textbf{Benchmark File} & \textbf{Category/Domain} & \textbf{Example Dataset} & \textbf{Target Dataset} \\
\midrule
1, 2 & BenchmarkT3-BucketLarge-1.tex & Daily Life \& Psychology (D1) & 45 & 450 \\
3, 4 & BenchmarkT3-BucketLarge-2.tex & History \& Geopolitics (D2) & 45 & 450 \\
5, 6 & BenchmarkT3-BucketLarge-3.tex & Markets \& Weather (D3) & 45 & 450 \\
7, 8 & BenchmarkT3-BucketLarge-4.tex & Disease \& Recovery / Medicine (D4) & 46 & 460 \\
9, 10 & BenchmarkT3-BucketLarge-5.tex & Economics \& Technology (D5) & 46 & 460 \\
11, 12 & BenchmarkT3-BucketLarge-6.tex & Environment \& Climate (D6) & 45 & 450 \\
13, 14 & BenchmarkT3-BucketLarge-7.tex & Law, Policy \& Ethics (D7) & 46 & 460 \\
15, 16 & BenchmarkT3-BucketLarge-8.tex & AI Safety \& Alignment (D8) & 45 & 450 \\
17, 18 & BenchmarkT3-BucketLarge-9.tex & Performance, Evaluation \& Luck (D9) & 46 & 460 \\
19, 20 & BenchmarkT3-BucketLarge-10.tex & Social Science \& Demographics (D10) & 45 & 450 \\
\bottomrule
\end{tabular}
\caption{Group assignments with benchmark files, categories, and dataset sizes}
\label{tab:group_assignments}
\end{table}

Note: Groups listed together (e.g., Groups 1 and 2) are cross-validation pairs that will swap datasets in Assignment 2.

\subsection{Cross-Validation for Assignment 2}

In Assignment 2, groups that worked on the same BenchmarkT3-BucketLarge file in Assignment 1 will \textbf{swap their datasets} for cross-validation:
\begin{itemize}[leftmargin=1.5em]
    \item Groups 1 and 2 will exchange their datasets
    \item Groups 3 and 4 will exchange their datasets
    \item Groups 5 and 6 will exchange their datasets
    \item Groups 7 and 8 will exchange their datasets
    \item Groups 9 and 10 will exchange their datasets
    \item Groups 11 and 12 will exchange their datasets
    \item Groups 13 and 14 will exchange their datasets
    \item Groups 15 and 16 will exchange their datasets
    \item Groups 17 and 18 will exchange their datasets
    \item Groups 19 and 20 will exchange their datasets
\end{itemize}

This cross-validation approach ensures that each group validates the work of another group on the same dataset, providing robust evaluation and learning opportunities.

\subsection{Group Formation}

Students will form groups by selecting their preferred group assignment through a Google Form. The form will allow students to choose which benchmark file and category they would like to work on. Each group will have approximately 6 students.

\textbf{Google Form Link:} \href{https://forms.google.com/your-form-link-here}{Group Selection Form}

Please complete the form by the deadline specified in the course schedule. Group assignments will be finalized based on student preferences and will be announced via the course platform.

\section{Student List}

The following table lists all students enrolled in the course. Student emails are formatted as Login ID + @stanford.edu.

\begin{longtable}{p{3cm}p{3cm}p{5cm}}
\toprule
\textbf{First Name} & \textbf{Last Name} & \textbf{Email} \\
\midrule
\endfirsthead
\toprule
\textbf{First Name} & \textbf{Last Name} & \textbf{Email} \\
\midrule
\endhead
\bottomrule
\endfoot
% Student entries from Studentlist.csv will be added here
% To generate the student list, run: python3 generate_student_list.py
% Then copy the output into this section
\midrule
\endlastfoot
% Student entries will be added here
\end{longtable}

\section{Assignment Files}

Each BenchmarkT3-BucketLarge file contains a collection of reasoning cases organized by domain and Pearl's Causality Hierarchy levels. The files are structured as follows:

\subsection{File Structure}

Each \texttt{BenchmarkT3-BucketLarge-*.tex} file contains:
\begin{itemize}[leftmargin=1.5em]
    \item \textbf{Bucket Overview:} Domain description, core themes, signature trap types, and case distribution
    \item \textbf{Pearl Level 1 Cases (Association):} Cases focusing on observational relationships
    \item \textbf{Pearl Level 2 Cases (Intervention):} Cases requiring understanding of interventions and causal effects
    \item \textbf{Pearl Level 3 Cases (Counterfactual):} Cases involving counterfactual reasoning
\end{itemize}

\subsection{Case Format}

Each case typically includes:
\begin{itemize}[leftmargin=1.5em]
    \item \textbf{Scenario:} A description of the situation or problem
    \item \textbf{Variables:} Key variables involved in the causal reasoning
    \item \textbf{Annotations:} Additional context or background information
    \item \textbf{Questions:} Reasoning questions to be answered
    \item \textbf{Expected Analysis:} The type of reasoning required
\end{itemize}

\section{Assignment Instructions}

\subsection{Objectives}

\begin{enumerate}[leftmargin=1.5em]
    \item Analyze the assigned BenchmarkT3-BucketLarge file
    \item Understand the causal reasoning challenges presented in each case
    \item Identify the types of reasoning required (Association, Intervention, Counterfactual)
    \item Apply the T³ architecture principles learned in class
    \item Document your analysis and findings
\end{enumerate}

\subsection{Deliverables}

Each group should prepare:
\begin{itemize}[leftmargin=1.5em]
    \item A comprehensive analysis of their assigned benchmark file
    \item Identification and classification of reasoning types in the cases
    \item Discussion of the causal reasoning challenges
    \item Application of T³ architecture concepts
    \item A summary report of findings
\end{itemize}

\subsection{Key Concepts to Apply}

When working on your assigned file, consider:
\begin{itemize}[leftmargin=1.5em]
    \item \textbf{Pearl's Causality Hierarchy:} Association $\rightarrow$ Intervention $\rightarrow$ Counterfactual
    \item \textbf{T³ Architecture:} Sycophancy and Skepticism mechanisms
    \item \textbf{Causal Reasoning:} Understanding cause-effect relationships
    \item \textbf{Confounding Variables:} Identifying and handling confounders
    \item \textbf{Selection Bias:} Recognizing and addressing selection issues
    \item \textbf{Collider Bias:} Understanding collider structures
    \item \textbf{Instrumental Variables:} Using instruments for causal inference
\end{itemize}

\section{Resources}

\subsection{Course Materials}

Refer to the following course materials:
\begin{itemize}[leftmargin=1.5em]
    \item Lecture slides on T³ Architecture for Sycophancy and Skepticism
    \item AGI Book, Volume \#2, Chapters 6 and 7
    \item Lecture on Pearl's Causality Hierarchy
    \item Assignment \#1 Specification (from Lecture 3)
\end{itemize}

\subsection{Additional Reading}

\begin{itemize}[leftmargin=1.5em]
    \item Multi-LLM Collaborative Intelligence (MACI), The Path to AGI
    \item Course readings on SocraSynth.com
\end{itemize}

\section{Submission Guidelines}

\begin{itemize}[leftmargin=1.5em]
    \item Submit your analysis and report according to the course deadlines
    \item Follow the formatting guidelines provided by the instructor
    \item Include proper citations and references
    \item Ensure all group members contribute to the assignment
\end{itemize}

\section{Important Dates}

\begin{itemize}[leftmargin=1.5em]
    \item \textbf{Assignment \#1 Out:} January 7, 2026 (Lecture 2)
    \item \textbf{Assignment \#1 Specification:} January 12, 2026 (Lecture 3)
    \item \textbf{Assignment \#1 Due:} January 14, 2026 (Lecture 4)
\end{itemize}

\section{Contact}

For questions about this assignment, please contact:
\begin{itemize}[leftmargin=1.5em]
    \item \textbf{Instructor:} Prof. Edward Y. Chang
    \item \textbf{Course Assistant:} Longling Gloria Geng
\end{itemize}

\vspace{0.5cm}

\noindent\textit{Good luck with your assignment!}

\end{document}

